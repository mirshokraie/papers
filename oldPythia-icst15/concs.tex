\section{Conclusions} \label{Sec:concs}

%\javascript's highly dynamic nature and its complex interaction with DOM make it difficult for the tester to achieve high coverage test suite. 
In this paper, we presented a technique  to automatically generate  test cases for \javascript applications at two complementary levels: (1) individual \javascript functions, (2) event sequences. 
Our technique is based on algorithms to maximize function coverage and minimize function states needed for efficient test generation.
We also proposed a method for effectively generating test oracles along with the test cases, for detecting faults in \javascript code as well as on the DOM tree. We implemented our  approach in an open-source tool called \tool. We empirically evaluated \tool on 13 web applications. The results show that the generated tests by \tool achieve high coverage (68.4\% on average),  and that the injected faults can be detected with a high accuracy rate (recall 70\%, precision 100\%). 
%In comparison, the other framework has a coverage of 45.3\%, and a recall of 10.5\% for the injected faults.
%
%Our future work will include improving the guided exploration technique to capture the behaviour of functions in terms of coverage as well as the execution frequency of function's statements, in addition to function coverage. Conducting larger experiments with \tool to obtain more general conclusions also forms part of our future work.
 