\section{Challenges and Motivation} \label{Sec:motivation}

\begin{figure}
%\medskip
\begin{lstlisting}
	var currentDim=20;
	function cellClicked() {	
		var divTag = '<div id='divElem' />';
		if($(this).attr('id') == 'cell0'){
			$('#cell0').after(divTag);
			$('div #divElem').click(setup);
		}
		else if($(this).attr('id') == 'cell1'){
			$('#cell1').after(divTag);
			$('div #divElem').click(function(){setDim(20)});
		}
	}
			 			
	function setup() {
		setDim(10);
		$('#startCell').click(start);
	}
	
	function setDim(dimension) {
		var dim=($('#endCell').width() + $('#endCell').height()))/dimension;
		currentDim += dim;
		$('#endCell').css('height', dim+'px');
		return dim;
	}
	
	function start() {
		if(currentDim > 40)
		  $(this).css('height', currentDim+'px');
		else $(this).remove(); 
	}
	
	$document.ready(function() {
		...
		$('#cell0').click(cellClicked);	
		$('#cell1').click(cellClicked);	
	});
\end{lstlisting}
\vspace{-0.1in} 

\caption{\javascript code of the running example.}
\label{Fig:funcCovgExample}
\vspace{-0.1in} 

\end{figure}
 

%The goal of our work is to automatically generate unit tests and oracles for detecting faults in \javascript applications. 
%Manually locating the source of error in \javascript is a daunting task in many different ways. 
%Automated test generation is challenging due to the event-driven dynamic nature of \javascript, and its extensive interaction with the DOM  resulting in many execution paths and states.
%and because there can be significant amounts of error masking 
In this section, we illustrate some of the challenges associated with test generation for \javascript applications. 
%

\figref{funcCovgExample} presents a snippet of a \javascript game application that we use as a running example throughout the paper. This simple example uses the popular jQuery library \cite{jquery-api} and contains four main \javascript functions: 
%\ali{before going to the details of the code, say something about the type/functionality of this application. Is this a game application? Same game?}

\begin{enumerate} [noitemsep]

\item \code{cellClicked} is bound to the event-handlers of DOM elements with IDs \code{cell0} and  \code{cell1} (Lines 34--35). These two DOM elements become available when the DOM is fully loaded (Line 32). Depending on the element clicked, \code{cellClicked} inserts a \code{div} element with ID \code{divElem} (Line 3) after the clicked element and makes it clickable by attaching either \code{setup} or \code{setDim} as its event-handler function (Lines 5--6, 9--10).    

\item \code{setup} calls \code{setDim} (Line 15) to change the value of the global variable \code{currentDim}. It further makes an element with ID \code{startCell} clickable by setting its event- handler to \code{start} (Line 16).

\item \code{setDim} receives an input variable. It performs some computations to set the \code{height} value of the \code{css} property of a DOM element with ID \code{endCell}  and the value of \code{currentDim} (Lines 20--22). It also returns the computed dimension.

\item \code{start} is called at runtime when the element with ID \code{startCell} is clicked (Line 16), which either updates the width dimension of the element on which it was called, or removes the element (Lines 27-29).
\end{enumerate}

There are four main challenges in testing \javascript applications. 
 
The first challenge is that a fault may not immediately propagate into a DOM-level observable failure. %become immediately apparent due to the event-driven nature of the execution. 
For example, if the `\code{+}' sign in Line 21 is mistakenly replaced by `\code{-}', the affected result does not immediately propagate to the observable DOM state after the function exits. While this mistakenly changes the value of a global variable, \code{currentDim}, which is 
later used in  \code{start} (Line 27), it neither affects the returned value of the \code{setDim} function nor the \code{css} value of element \code{endCell}. Therefore, a GUI-level event-based testing approach does not help to detect the fault in this case.
%Considering the complex interactions between various \javascript functions, 

The second challenge is related to fault localization; even if the fault propagates to a future DOM state and a DOM-level test case detects it, finding the actual location of the fault is still challenging for the tester as the DOM-level test case is agnostic of the \javascript code.
However, a unit test case that targets individual functions, e.g., \code{setDim} in this running example, helps a tester to spot the fault, and thus easily resolve it. %This is the reason we believe  generating unit tests with oracles in this paper.

The third challenge pertains to the event-driven dynamic nature of \javascript, and its extensive interaction with the DOM  resulting in many state permutations and execution paths. In the initial state of the example, clicking on \code{cell0} or \code{cell1} takes the browser to two different states as a result of the \code{if-else} statement in Lines 4 and 8 of the function \code{cellClicked}. 
Even in this simple example, expanding either of the resulting states has different consequences due to different functions that can be potentially triggered. 
Executing either \code{setup} or \code{setDim} in Lines 6 and 10 results in different execution paths, DOM states, and code coverage. 
It is this dynamic interaction of the \javascript code with the DOM (and indirectly CSS) at runtime that makes it challenging to generate test cases for \javascript applications.  

The fourth important challenge in unit testing \javascript functions that have DOM interactions, such as \code{setDim}, is that the DOM tree in the state expected by the function, has to be present during unit test execution. Otherwise the test will fail due to a \code{null} or \code{undefined} exception. This situation arises often in modern web applications that have many DOM interactions.
%In the following sections, we describe how we automatically generate \javascript unit tests that target (1) functions, and (2) DOM events to achieve high coverage.  
%\ali{Challenges: unit testing JS code that has DOM API calls is challenging. We need to have the DOM tree available (as input) otherwise the test fails.}