\section{Extensions w.r.t. the ICST'15 paper}
(This appendix is for reviewers only, and will not be included in the final version).\\

This paper is a revised and extended version of our paper `\tool: Automated JavaScript Unit Test Generation',  which appeared in the \emph{Proceedings of the International Conference on Software Testing, Verification and Validation (ICST 2015)} \cite{mirshokraie:icst15}.
%It is worth mentioning that this paper received the \emph{Best Paper Runner-up Award} at ICST'13.\\

The main extensions include:

\begin{itemize}

\item Improving the approach as follows:

\begin{enumerate}
\item Revising the function state reduction technique (explained in \secref{jsFuncTest}); The approach diagram (\figref{approach-view}), function state reduction algorithm (\algref{stateAbstractionAlgo}), and the description of unit test generation in \secref{approach} is revised to reflect changes to the state reduction technique. 
\end{enumerate}

\item Extending the evaluation section as follows:

\begin{enumerate} 
\item Comparing the fault finding capability obtained by (1) using the state reduction mechanism and (2) including the whole application's state.
\item Comparing the number of assertions in the following scenarios:
(1) capturing the whole application's state, (2) with enabling the state reduction technique only, and (3) with applying the state reduction as well as the mutation-based oracle generation algorithm;
\end{enumerate}

\item Adding one more research question to the evaluation section (RQ3 in \secref{evaluation}) to discuss the results of the new comparisons. 

\item The new results are presented in \secref{results}. The comparative results are presented in \tabref{efficiency-abs-mut-table} and \tabref{faultDetection-table}.
\item Adding a discussion section (\secref{discussion}). 
\item A more elaborate coverage of related work (\secref{related}).
\item Furthermore, we made a full pass over the text,
   leading to many improvements throughout the paper,
   including various changes that originated from discussions at the
   ICST'15 conference and other venues where this work was presented. 

\end{itemize}

\medskip

While difficult to measure objectively, we believe this accounts for
   an extension of over 30\% -- a common guideline used for journal versions
   of papers that are based on conference papers.
