\subsection{DOM-level event-based testing}
\label{Sec:domTest}

In the second step, our technique first extracts sequences of events from the inferred state-flow graph. These sequences of events are used in our test case generation process.
%We generate test cases at two complementary levels, as described below. 

\headbf{DOM-level event-based test cases} %\label{Sec:DOMEventTesting}
To verify the behaviour of the application at the user interface level, each event path, taken from the initial state (\code{Index}) to a leaf node in the state-flow graph, is used to generate DOM event-based test cases. 
Each extracted path is converted into a \junit \selenium-based test case, which executes the sequence of events, starting from the initial DOM state. %The generated JUnit test case is able to trigger events by using the information included in the state flow graph such as the element on which the event is triggered as well as  type of the fired event.
Going back to our running example, one possible event sequence to generate is: \code{\$(`\#cell0').click$\rightarrow$\$(`div \#divElem').click$\rightarrow$\$(`\#st\-artCell').click}. 

To collect the required trace data, we capture all DOM elements and their attributes after each event in the test path is fired. This trace is later used in our DOM oracle comparison, as explained in \secref{oracleGen}. 


 
 

  
     