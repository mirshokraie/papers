\section{Tool Implementation}
\label{Sec:tool}
We have implemented our \javascript regression testing approach in a tool called \jsart.
\jsart is written in Java and is available for download.\footnoteremember{download}{\url{http://salt.ece.ubc.ca/content/jsart/}}
%\jsart is implemented in Java and operates as a plugin on top of 
%\crawljax \cite{mesbah:icwe08} \cite{mesbah:tse12}.

%\jsart has three modes:
%(1) Invariant generation mode, in which traces are collected and invariants are automatically %detected, (2) Filtering mode, in which unstable invariants are eliminated,
%and (3) Testing mode, during which the inferred invariant assertions are used for checking the runtime behaviour.

\jsart extends and builds on top of our InvarScope \cite{groenevel:tech10} tool. For \javascript code interception, we use an enhanced version of Web-Scarab's proxy \cite{bezemer:esec09}. 
This enables us to automatically analyze and modify the content of HTTP responses before they reach the browser. To instrument the intercepted code, Mozilla Rhino\curl{http://www.mozilla.org/rhino/} is used to parse \javascript code to an AST, and back to the source code after instrumentation. The AST generated by Rhino's parser has 
traversal API's, which we use to search for program points where instrumentation code needs to be added. 
For the invariant generation step, we have extended Daikon \cite{ernst2007daikon} with support for accepting input and generating output in \javascript syntax. The input files are created from the trace data and fed through the enhanced version of Daikon to derive dynamic invariants.
The navigation step is automated by making \jsart operate as a plugin on top of our dynamic \ajax crawler, \crawljax \cite{mesbah:tweb11}.\curl{http://www.crawljax.com}




%\jsart is implemented in Java and operates as a plugin on top of 
%\crawljax \cite{mesbah:icwe08} \cite{mesbah:tse12}.

%\jsart has three modes:
%(1) Invariant generation mode, in which traces are collected and invariants are automatically %detected, (2) Filtering mode, in which unstable invariants are eliminated,
%and (3) Testing mode, during which the inferred invariant assertions are used for checking the runtime behaviour.
