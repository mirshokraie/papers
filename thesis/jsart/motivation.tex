\section{Motivation and Challenges}
\label{Sec:motivation}

\begin{figure}[t]
\medskip
\begin{lstlisting}
	function setDim(height, width) {		
		var h = 4*height, w = 2*width;		
		...
		return{h:h, w:w};
	}
	
	function play(){
		$(#end).css("height", setDim($('body').width(), $('body').height()).h + 'px');		
		...
	}	
\end{lstlisting}
\vspace{0.1in} 
\mycaption{Motivating \javascript example.}
\label{Fig:motivating_example}
\end{figure}
%\subsection{Challenges}
%\label{challenges}

%In this section we provide a simple example from a real \javascript application. The example shows how prevelant errors can happen in applications.  
\figref{motivating_example} shows a simple \javascript code snippet. %taken from a real-world web application and modified for simplicity.
%We modified the real sample to make the code readable.
Our motivating example consists of two functions, called \code{setDim} and \code{play}. The \code{setDim} function has two parameters, namely \code{height} and \code{width}, with a 
simple mathematical operation (Line 2). The function returns local variables, \code{h} and \code{w} (Line 4). \code{setDim} is called in the \code{play} function (Line 8) to set the \code{height} value of the CSS property of the DOM element with ID \code{end}.
Any modification to the values of \code{height} or \code{width} would influence the returned values of \code{setDim} as well as the property of the DOM element.
Typical programmatic errors include swapping the order of \code{height} and \code{width} when they are respectively assigned to local variables \code{h} and \code{w} or calling \code{setDim} with wrong arguments, i.e., changing the order of function arguments. 

Detecting such regression errors is a daunting task for web developers, especially in programming languages such as \javascript, which are known to be challenging to test.
One way to check for these regressions is to define invariant expressions of expected behaviour \cite{Meyer:1992} over program variables and assert their correctness at runtime. 
This way any modification to \code{height}, \code{width}, \code{h}, or \code{w} that violates the invariant expression will be detected. However, manually expressing such assertions for web applications with thousands of lines of \javascript code and several DOM states, is a challenging and time-consuming task. Our aim in this work is to provide a technique that automatically captures regression faults through generated \javascript  assertions.

