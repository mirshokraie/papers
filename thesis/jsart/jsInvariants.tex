\section{JavaScript Invariants}
\label{Sec:jsInvariants}
In this section, we focus on our approach for automatically detecting dynamic JavaScript program 
invariants by tracing and analyzing client-side variables as well as JavaScript DOM modifications. We discuss how
those invariants are automatically injected as assertions in the application's JavaScript source code.

\subsection{JavaScript Instrumentation}
In order to log JavaScript variables, we modify JavaScript source code on-the-fly to add instrumentation code. 
We intercept all the JavaScript code, both in JavaScript files and HTML pages,
which passes from the server to the browser using a proxy\cite{?}. We first parse the intercepted source code into 
an Abstract Syntax Tree (AST). We then traverse the AST in search of interesting program points to add instrumentation
code. We instrument program variables as well as DOM modifications as we describ below. Once the code is instrumented, we
serialize the AST back to the corresponding source code file and pass it to the browser.

{\bf JavaScript DOM Modifications.} One type of trace data which we include in the execution trace is how certain
DOM elements and their attributes are modified at runtime through JavaScript. For instance, by tracing how a {\bf class}
attribute of an anchor (i.e. {\bf A}) element is changed during various execution runs, we can infer the range of values for
the class attribute of the anchor tag. Based on our observations, JavaScript DOM modifications usually follow a certain pattern.
Once the pattern is reverse engineered, we can add proper instrumentation code around the pattern to trace the changes.
In patterns that we observed, first a JavaScript API is used to find the desired DOM element. Next, a function is called on the returned object
which is responsible for the actual modification of the DOM-tree.
After recognizing a pattern in the parsed AST, we add instrumentation code that records the value of the
DOM attribute before and after the actual modification. Therefore, we are able to trace DOM modifications that happen programmatically through JavaScript.

\figref{example_dom_instrumentation} shows the automatically instrumented JavaScript code for DOM manipulations. The {\bf save} function,
is called at each program point with a new array argument, which has a program point identifier as the first element, 
followed by a number of arrays for the variables. These arrays contain the name, type, and value of the variables. 
The {\bf addvariable} function finds out the types of the variables passed by the save function at runtime and creates 
the arrays. The {\bf height} value of the {\bf css} style property is traced before and after the DOM modification. Note that apart
from the DOM modification patterns, reading the value of a DOM element property is also done in different patterns by
various JavaScript libraries. For instance, in the case of jQuery, the number of arguments passed to the attr function 
is determining whether we are setting the property or getting the current value.

{\bf JavaScript Program Variables.} We search for function entry and function exit points through identifying function
definitions in the AST and injecting statements at the start, end, and before every return statement. We instrument the
code to monitor value changes of global variables, function arguments, and local variables.
Per program point, we record information on script name, the function name, and the line number which are used for debugging
purposes. for each variable we collect information on name, run-time type and the actual values. The runtime type is stored
because JavaScript is a loosely typed language, i.e. the types of variables cannot be determined syntactically, thus
we log the variable types at runtime.

\figref{example_var_instrumentation} shows the automatically instrumented JavaScript code for program local variables as well as function arguments.
Similar to DOM modifications, we trace variables and function arguments through calling {\bf save} function. 
In order to trace value changes of local variables, we instrument the JavaScript code at exit point of the function. 
However, to keep track of function arguments as well as global variables, we instrument the code at both entry and exit point of the function.  



\begin{figure}
\medskip
\begin{lstlisting}
	function playIt() {
		...	
		/* Save the attribute value before manipulation */
		save (new Array('GhostBusters.js:playIt:POINT80', addVariable("$('#end').css('height')", $('#end').css('height')));
	
		/* jQuery DOM manipulation */
		$(#end).css("height", (getDim().h) + 'px');
	
		/* Save the attribute value after manipulation */
		save (new Array('GhostBusters.js:playIt:POINT81', addVariable("$('#end').css('height')", $('#end').css('height')));
		...
	}	

\end{lstlisting}
\caption{Instrumented JavaScript code for tracing JavaScript DOM modifications.}
\label{Fig:example_dom_instrumentation}
\end{figure}




\begin{figure}
\medskip
\begin{lstlisting}
	function getDim(height, width) {
		
		/* Save function arguments at entry point of the function */
		save (new Array('GhostBusters.js:getDim:ENTER:POINT50', addVariable("height", height)));
		save (new Array('GhostBusters.js:getDim:ENTER:POINT50', addVariable("width", width))); 

		var h = height, w = width;		
 		h = h*2;
 		w = w*3;		
 		
		/* Save variables and function arguments before the function returns */
		save (new Array('GhostBusters.js:getDim:EXIT:POINT50', addVariable("height", height)));
		save (new Array('GhostBusters.js:getDim:EXIT:POINT50', addVariable("width", width)));
		save (new Array('GhostBusters.js:getDim:EXIT:POINT50', addVariable("h", h)));
		save (new Array('GhostBusters.js:getDim:EXIT:POINT50', addVariable("w", w)));
	
		return {h:h, w:w}
	}	

\end{lstlisting}
\caption{Instrumented JavaScript code for tracing program variables and function arguments.}
\label{Fig:example_var_instrumentation}
\end{figure}

  


\begin{figure}
\medskip
\begin{lstlisting}
	function playIt() {
		...	
		/* Checking assertion before manipulation. Add an entry to the assertionFailure, if it is violated. */
		if ($('#end').css('height') != 40)
  		  window.assertionFailure.push(new Array("$('#end').css("height") == 40", "GhostBusters.js:playIt:POINT80")); 
	
		/* jQuery DOM manipulation */
		$(#end).css("height", (getDim().h) + 'px');
	
		/* Checking assertion after manipulation. Add an entry to the assertionFailure, if it is violated. */
		if ($('#end').css('height') != 200 && $('#end').css('height') != 400)
  		  window.assertionFailure.push(new Array("$('#end').css('height') == 200 || $('#end').css('height') == 400", 											  
  		  "GhostBusters.js:playIt:POINT81")); 	
  	...
	}	

\end{lstlisting}
\caption{Invariant assertion code for JavaScript DOM modifications.}
\label{Fig:example_dom_assertion}
\end{figure}


\begin{figure}
\medskip
\begin{lstlisting}
	function getDim(height, width) {
		
		/* Checking assertions at function entry point. Add an entry to the assertionFailure, if it is violated. */
		if (width <= height) 
  		  window.assertionFailure.push(new Array('width > height', 'GhostBusters.js:getDim:ENTER:POINT50'));		
		
		var  h = height, w = width;					
 		h = h*2;
 		w = w*3;	
		
		/* Checking assertions before the function returns. Add an entry to the assertionFailure, if it is violated. */
		if (width <= height)
  		  window.assertionFailure.push(new Array('width > height', 'GhostBusters.js:getDim:EXIT:POINT50'));   	
		if (w <= h) 
  		  window.assertionFailure.push(new Array('w > h', 'GhostBusters.js:getDim:EXIT:POINT50'));	   	
	
		return {h:h, w:w}
	}	

\end{lstlisting}
\caption{JavaScript invariant assertion code for program variables and function arguments.}
\label{Fig:example_var_assertion}
\end{figure}

\subsection{Logging Execution Traces}
Once the JavaScript code is instrumented, we need to run the application to produce an execution trace. 
We automate this step by using our AJAX crawling tool CRAWLJAX \cite{?}, which opens the web application
in a browser, detects possible state transitions on the DOM-tree, and clicks through the state space.

Next step is saving the trace data that is being produced at runtime as we navigate through the various
states of the web application.
Keeping the trace data in the browser's memory during the program execution can make the browser very slow when a
huge trace is produced. On the other hand, sending data item to the server as soon as the item is generated, can 
put a heavy load on the server, due to the huge amounts of data and high frequency of HTTP requests.
In order to tackle the aforementioned challenges, in our approach we buffer a certain amount of trace data in the 
memory in an array, post the data as an HTTP request to a proxy server when the buffer's size reaches a predefined threshold,
and immediately clear the buffer in the browser's memory afterwards. Since the data arrives at the server in 
a synchronized manner, we concatenate the tracing data into a single trace file on the server side. 
This way the, the extra work that the browser and server have to conduct for our tracing step becomes manageable.

\subsection{Deriving JavaScript Invariants}
In order to infer JavaScript invariants, we automatically analyze the obtained trace data and generate 
input files for Daikon \cite{?}. We have extended Daikon with support for accepting input and generating 
output in JavaScript syntax. The input files are fed to Daikon to derive likely invariants.

\subsection{Injecting Invariants as Assertions}
Once the initial invariants are derived, we automatically detect unstable invariants and filter them as described earlier. Then
from each invariant, we generate an assertion in JavaScript format. We use on-the-fly transformation to inject the assertions 
directly into the JavaScript code in the browser. Since we have all the information about the program points and the location 
of the invariants, we can inject the assertions at the correct position in the JavaScript source code through a proxy, as 
they are served by the server. This way the assertions gain access to all variables needed at runtime.
Once the assertions are in place, the web application is ready for regression testing a different version of the original application. 
The actual testing can be done by manually clicking through the application, running an existing test suite, or using an automated crawler.
Failed assertions during testing phase are used to generate a test report after the test execution session.

\figref{example_dom_assertion} shows the automatically injected invariant assertions, checking the {\bf css} property of the example shown in
\figref{example_dom_instrumentation}. Assertions are checked before and after the DOM modification, against the {\bf height} value of {\bf css} property which
is derived from the corresponding invariants. Similarly, in \figref{example_var_assertion}, we depict injected assertions which are checking 
value of variables and function arguments of the example shown in \figref{example_var_instrumentation}. In this example, inferred invariants yield information
about the inequality relation between function arguments, {\bf width} and {\bf height}, as well as local variables, {\bf w} and {\bf h}. Assertions check 
the corresponding ineqaulities to be hold during the application run-time.  
