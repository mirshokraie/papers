\chapter{Conclusion} \label{Chap:conc}
\javascript is increasingly being used to create modern interactive web applications that offload a considerable amount of their execution to the client-side. \javascript is a notoriously challenging language for web developers to use, maintain, analyze and test. This dissertation has focused on exploring strategies for testing \javascript-based web applications.
\section{Contributions}
The main contributions of the thesis can be summarized as follows:
\begin{itemize}
\item A new automated technique for \javascript regression testing, which is based on dynamic analysis to infer invariant assertions; The obtained assertions are injected back into the \javascript code to uncover regression faults in subsequent revisions of the web application under test. 
\item The first \javascript mutation testing tool, which is capable of guiding the mutation generation towards behaviour-affecting mutants in error-prone portions of the code; The mutation testing method combines dynamic and static analysis to mutate branches that are within highly ranked functions and exhibit high structural complexity.
\item An automatic technique to generate test cases for \javascript functions and events; We use a mutation-based algorithm to effectively generate test oracles, capable of detecting regression \javascript and DOM-level faults. The technique uses a combination of function converge maximization and function state abstraction algorithms to efficiently generate unit test cases.
\item Exploiting an existing DOM-based test suite to generate unit-level assertions for applications that highly interact with the DOM through the underlying \javascript code; We utilize
existing DOM-dependent assertions as well as useful execution information inferred from a DOM-based test suite to automatically generate assertions used for testing individual \javascript functions.
\end{itemize}