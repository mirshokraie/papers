\chapter{Conclusions} \label{Chap:conc}
\javascript is increasingly being used to create modern interactive web applications that offload a considerable amount of their execution to the client-side. \javascript is a notoriously challenging language for web developers to use, maintain, analyze and test. %This thesis has focused on exploring strategies for testing \javascript-based web applications.
The work presented in this dissertation aims at improving the state-of-the-art in testing \javascript web applications by proposing a new set of techniques and tools.
\section{Contributions}
%The main contributions of the thesis can be summarized as follows:
The main contributions of the thesis are as follows: 
\begin{itemize}
\item A new automated technique for \javascript regression testing, which is based on dynamic analysis to infer invariant assertions; The obtained assertions are injected back into the \javascript code to uncover regression faults in subsequent revisions of the web application under test. 
\item The first \javascript mutation testing tool, which is capable of guiding the mutation generation towards behaviour-affecting mutants in error-prone portions of the code; The mutation testing method combines dynamic and static analysis to mutate branches that are within highly ranked functions and exhibit high structural complexity.
\item An automatic technique to generate test cases for \javascript functions and events; We use a mutation-based algorithm to effectively generate test oracles, capable of detecting regression \javascript and DOM-level faults. The technique uses a combination of function converge maximization and function state abstraction algorithms to efficiently generate unit test cases.
\item Exploiting an existing DOM-based test suite to generate unit-level assertions for applications that highly interact with the DOM through the underlying \javascript code; We utilize
existing DOM-dependent assertions as well as useful execution information inferred from a DOM-based test suite to automatically generate assertions used for testing individual \javascript functions.
\end{itemize}
\section{Revisiting Research Questions} 
In the beginning of this thesis, we designed two research questions. We believe that the contributions show that we have addressed the research questions.
\headbf{Research Question 1}

\emph{How can we generate effective test cases for \javascript web applications?}

In order to answer the first research question, \chapref{jsart} targets web application testing from the invariant assertions points of view. These invariants formulate the main characteristics of the application under test that will remain unchanged as the application evolves. Therefore, these type of assertions can be used towards regression testing. The empirical study on nine open source \javascript applications show that the proposed approach is able to effectively infer stable assertions and 
detect regression faults with minimal performance overhead.

Our invariant generation technique is based on the assumption that the program specifications are not changed frequently in subsequent revisions. However, if major changes affect the core properties of the application, the inferred invariants from the original version may not be valid any longer. Moreover, unlike post-condition assertions, invariant-based ones are inline assertions, which are checked at different points of the program's code. Therefore, it can become difficult for the tester to comprehend these type of assertions.

In order to generate oracles that can be used during the common testing cycles of a large system, including unit and GUI testing, we proposed \jseft in \chapref{jseft}. \jseft generates test cases combined with post-condition assertions at the two complementary levels of unit and event-based tests. We use mutation testing to produce our assertions. 
To evaluate the effectiveness of \jseft we consider a state-of-the-art \javascript test generation framework as a basis to compare our technique. The results of the empirical evaluation indicate that the approach generates test cases with high fault finding capability. 

Further analysis of the results revealed that (1) although, the generated assertions by \jseft are effective in detecting the injected faults, the use of mutation testing for the purpose of assertion generation can negatively impact the performance, and (2) event-based tests can potentially miss the code-related errors (32\% on average) if the fault does not propagate to the observable GUI state. We observed that the rate of missed faults are higher for the applications that have tight interaction with the GUI through the underlying executable code. These two observations form the basis of \chapref{atrina}, where we make use of the GUI-dependent assertions as a guide to generate code related unit-level assertions.

In \chapref{atrina}, we proposed \atrina which utilizes the existing GUI-based (\ie DOM) assertions as well as useful execution information inferred from a GUI test suite to automatically generate assertions used for testing individual functions. This work is currently under preparation, however, the initial results confirm that the generated unit-level assertions surpass the fault finding capability of DOM-based assertions with 37\% on average. We also found out that \atrina outperforms mutation-based assertion generation technique in terms of the time efficiency.

During the evaluation of different test generation techniques proposed in this thesis, we realized that using hidden scopes (\ie function closures) is quite popular in writing \javascript applications. 
Hidden scopes in \javascript language provide a way to make variables and functions private, thus keeping them out of the global scope.
While function closures can be called during the testing process at the highest program scope they belong to, it is not possible to call them directly in test cases, which makes it challenging to assess their outcomes.
One possible future direction is to measure the extent of such hard-to-test code written by developers by conducting a thorough empirical study.
The results of the study can be used towards generating effective test cases by identifying hard-to-test scopes, and if possible expose them to the testing unit through automated code refactoring. \javascript developers can also make use of the results of empirical study as a coding recommendation to make their future applications more testable and consequently more maintainable.

\headbf{Research Question 2}

\emph{How can we effectively assess the quality of the written test suites for \javascript applications?}

To address the second research question, in \chapref{mutandis} we proposed \mutandis, a generic mutation testing approach, that guides the mutation generation towards error-prone sections of the program. The empirical evaluation indicates that \mutandis can (1) significantly reduce the number of equivalent mutants, and (2) guide testers towards designing test cases for important portions of the code from the application's behaviour point of view. 

Reducing the number of equivalent mutants can potentially eliminate stubborn (hard-to-kill) mutants, which are particularly important for assessing the fault finding capability of test cases. The current evaluation results show that \mutandis does not negatively influence the stubbornness of the mutants.
However, our approach is not particularly designed to generate such mutations.
We found out that the stubbornness of the mutants generated by \mutandis stems from the inherent characteristics of the \javascript functions, such as function variadicity. Therefore, one interesting future work direction is to enhance the mutation generation technique by taking into account such particular function characteristics. This way we can reduce the number of equivalent mutants while increasing the number of stubborn mutants.

Another future work is to consider GUI-level mutations, which are particularly designed to assess the quality of GUI-level tests. However, this is involved with a number of challenges. Since in GUI mutation, the output is the resulting state from an executed event, the scope of the mutation operators differs from the traditional code-based mutant generators. Therefore, we need to define (1) a new set of GUI-based mutation operators, and (2) a new equivalent mutant detection technique.
  



