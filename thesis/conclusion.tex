\chapter{Conclusions} \label{Chap:conc}
\javascript is increasingly being used to create modern interactive web applications that offload a considerable amount of their execution to the client-side. \javascript is a notoriously challenging language for web developers to use, maintain, analyze and test. %This thesis has focused on exploring strategies for testing \javascript-based web applications.
The work presented in this dissertation aims at improving the state-of-the-art in testing \javascript web applications.
%\section{Contributions}
%The main contributions of the thesis can be summarized as follows:
%The main contributions of the thesis are as follows: 
%\begin{itemize}
%\item A new automated technique for \javascript regression testing, which is based on dynamic analysis to infer invariant assertions; The obtained assertions are injected back into the \javascript code to uncover regression faults in subsequent revisions of the web application under test. 
%\item The first \javascript mutation testing tool, which is capable of guiding the mutation generation towards behaviour-affecting mutants in error-prone portions of the code; The mutation testing method combines dynamic and static analysis to mutate branches that are within highly ranked functions and exhibit high structural complexity.
%\item An automatic technique to generate test cases for \javascript functions and events; We use a mutation-based algorithm to effectively generate test oracles, capable of detecting regression \javascript and DOM-level faults. The technique uses a combination of function converge maximization and function state abstraction algorithms to efficiently generate unit test cases.
%\item Exploiting an existing GUI-based (\ie DOM) test suite to generate unit-level assertions for applications that highly interact with the GUI through the underlying \javascript code; We utilize
%existing assertions as well as useful execution information inferred from a GUI test suite to automatically generate assertions used for testing individual \javascript functions.
%\end{itemize}
\section{Revisiting Research Questions} 
In the beginning of this thesis, we designed two research questions. We believe that the contributions show that we have addressed the research questions.
\subsection{Research Question 1}

\emph{How can we generate effective test cases for \javascript web applications?}

To this end, we proposed a set of techniques to automatically test the correct behaviour of the application. Mutation analysis has subsequently been used to assess the effectiveness of the generated tests. 

\headbf{\chapref{jsart}} We proposed \jsart, which targets web application testing from the invariant assertions points of view. Program invariants formulate the main characteristics of the application under test that remain unchanged as the application evolves. Therefore, they can be used towards regression testing. Our technique first infers such invariants from the application, and then convert them to test assertions. Our empirical study shows that the proposed approach is able to infer stable assertions and accurately detect regression faults. Our technique is involved with minimal performance overhead, and thus it may also scale well for industrial applications in practice.  Moreover, manual detection of invariants is a time consuming task. Therefore, using our tool can reduce the manual effort involved with inferring invariants. 

However, our invariant generation technique is based on the assumption that the program specifications are not changed frequently in subsequent revisions. If major changes affect the core properties of the application, the inferred invariants from the original version may not be valid any longer. Furthermore, it may not be possible to convert every program's characteristic into a useful invariant. For instance, game applications are usually involved with huge amount of state changes as the application executes. In such cases, the application may contain only a few invariants. 

\headbf{\chapref{jseft}} In order to generate a more generic type of oracles that can be used during the testing cycle of various applications, including unit and GUI testing, we proposed \jseft in \chapref{jseft}. \jseft generates test cases combined with post-condition assertions at the two complementary levels of unit and event-based tests. We use mutation testing to produce our assertions. 
To evaluate the effectiveness of \jseft we consider a state-of-the-art \javascript test generation framework as a basis to compare our technique. The results of the empirical evaluation indicate that the approach generates test cases with high fault finding capability. In addition to reducing the manual effort required for producing test and assertions, using our tool makes it easier for the tester to find the root cause of the error through the fine grained unit assertions. Moreover, our tool provides a better test suite comprehension as we reduce the number of test cases. Note that the generated tests can become hard to understand because of the huge number of test cases and assertions.

Though the evaluation results points to the effectiveness of \jseft in detecting faults, we found out that the generated event-based tests are brittle. Therefore, if trivial changes occur on the GUI of the application, our event-based assertions may not be valid anymore. A more robust mutation-based assertion generation is required to address this problem.
Further analysis of the results obtained from \jseft revealed that (1) although, the generated assertions by \jseft are effective in detecting the injected faults, the use of mutation testing for the purpose of assertion generation can negatively impact the performance, and (2) event-based tests can potentially miss the code-related errors if the fault does not propagate to the observable GUI state. We observed that the rate of missed faults is higher for the applications that have tight interaction with the GUI through the underlying executable code. These two observations form the basis of \chapref{atrina}, where we make use of the GUI-dependent assertions as a guide to generate code related unit-level assertions.

\headbf{\chapref{atrina}} Fruitful observations from analyzing the results of \jseft led us to propose \atrina, which utilizes the existing GUI-based (\ie DOM) assertions as well as useful execution information inferred from a GUI test suite to automatically generate assertions used for testing individual functions. This work is currently under preparation, however, the initial results confirm that the generated unit-level assertions surpass the fault finding capability of DOM-based assertions. We also found out that \atrina outperforms mutation-based assertion generation technique in terms of the time efficiency.

\headbf{Further Observations} During the evaluation of different test generation techniques proposed in this thesis, we realized that using function closures is quite popular in writing \javascript applications. 
Function closures in \javascript language provide a way to make variables and functions private, thus keeping them out of the global scope.
While function closures can be called during the testing process at the highest program scope they belong to, it is not possible to call them directly in test cases, which makes it challenging to assess their outcomes.
One possible future direction is to measure the extent of such hard-to-test code written by developers by conducting a thorough empirical study.
The results of the study can be used towards generating effective test cases by identifying hard-to-test scopes, and if possible expose them to the testing unit through automated code refactoring. \javascript developers can also make use of the results of empirical study as a coding recommendation to make their future applications more testable and consequently more maintainable.

\subsection{Research Question 2}

\emph{How can we effectively assess the quality of the existing \javascript test cases?}

We used mutation analysis as the test assessment technique. In the following, we explain our observations as well as potential future works to further improve and expand the use of mutation analysis. 

\headbf{\chapref{mutandis}} To assess the quality of test cases, we proposed \mutandis, a generic mutation testing approach, that guides the mutation generation towards behaviour-affecting mutants in error-prone portions of the code. The empirical evaluation indicates that \mutandis can (1) significantly reduce the number of equivalent mutants, and (2) guide testers towards designing test cases for important portions of the code from the application's behaviour point of view. 

One of the main challenges in adopting mutation testing in industrial environments, is the time and manual effort involved with detecting equivalent mutants. According to a recent study \cite{madeyski:tse13}, it takes 15 minutes on average to manually detect an equivalent mutant. While, there is still room for further reduction of equivalent mutants, our current evaluation results show that our approach considerably reduces the number of such useless mutants. This indicate that \mutandis can potentially reduce the effort required for eliminating such useless mutants, though this needs to be investigated by a thorough user study in future.
Testers can also use \mutandis to assess and compare the adequacy of their web application testing techniques. It is worth mentioning that while our approach is specifically designed for the purpose of mutation analysis, our proposed $FunctionRank$ metric tool, which measures the relative importance of functions can be used in program comprehension analysis as well.  

\headbf{Stubborn Mutants} Reducing the number of equivalent mutants can potentially eliminate stubborn (hard-to-kill) mutants, which are particularly important for assessing the fault finding capability of test cases. The current evaluation results show that \mutandis does not negatively influence the stubbornness of the mutants.
However, our approach is not particularly designed to generate such mutations.
We found out that the stubbornness of the mutants generated by \mutandis stems from the inherent characteristics of the \javascript functions. For example, one such characteristic is function variadicity, meaning that a function can be called with an arbitrary number of arguments. 
Therefore, one interesting future work direction is to enhance the mutation generation technique by taking into account such specific function features. This way we can reduce the number of equivalent mutants while increasing the number of stubborn mutants.

\headbf{GUI-level Mutation} One future direction is to consider GUI-level mutations for the purpose of assessing the quality of GUI-level tests. However, in GUI mutation the output is the resulting state from an executed event, and as a result the scope of the mutation operators differs from the traditional code-based mutant generators. Therefore, to make GUI-level mutation a useful test assessment technique, (1) a new set of GUI-based mutation operators should be defined, and (2) a new equivalent mutant detection method needs to be designed.
\section{Concluding Remarks}
The work presented in this thesis has focused on providing \javascript applications with a rich set of new test automation techniques. Given the growing popularity of \javascript and the challenges of testing this dynamic language, we see many opportunities for using the proposed techniques in practice. Further, developers can use our approaches not only to test the applications, but also to assess the adequacy of their web application tests. 
  



