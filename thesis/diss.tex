%%%%%%%%%%%%%%%%%%%%%%%%%%%%%%%%%%%%%%%%%%%%%%%%%%%%%%%%%%%%%%%%%%%%%%
% Template for a UBC-compliant dissertation
% At the minimum, you will need to change the information found
% after the "Document meta-data"
%
%!TEX TS-program = pdflatex
%!TEX encoding = UTF-8 Unicode

%% The ubcdiss class provides several options:
%%   gpscopy (aka fogscopy)
%%       set parameters to exactly how GPS specifies

%%         * single-sided
%%         * page-numbering starts from title page
%%         * the lists of figures and tables have each entry prefixed
%%           with 'Figure' or 'Table'
%%       This can be tested by `\ifgpscopy ... \else ... \fi'
%%   10pt, 11pt, 12pt
%%       set default font size
%%   oneside, twoside
%%       whether to format for single-sided or double-sided printing
%%   balanced
%%       when double-sided, ensure page content is centred
%%       rather than slightly offset (the default)
%%   singlespacing, onehalfspacing, doublespacing
%%       set default inter-line text spacing; the ubcdiss class
%%       provides \textspacing to revert to this configured spacing
%%   draft
%%       disable more intensive processing, such as including
%%       graphics, etc.
%%

% For submission to GPS
\documentclass[gpscopy,onehalfspacing,11pt]{ubcdiss}

% For your own copies (looks nicer)
% \documentclass[balanced,twoside,11pt]{ubcdiss}

%%%%%%%%%%%%%%%%%%%%%%%%%%%%%%%%%%%%%%%%%%%%%%%%%%%%%%%%%%%%%%%%%%%%%%
%%%%%%%%%%%%%%%%%%%%%%%%%%%%%%%%%%%%%%%%%%%%%%%%%%%%%%%%%%%%%%%%%%%%%%
%%
%% FONTS:
%% 
%% The defaults below configures Times Roman for the serif font,
%% Helvetica for the sans serif font, and Courier for the
%% typewriter-style font.  Configuring fonts can be time
%% consuming; we recommend skipping to END FONTS!
%% 
%% If you're feeling brave, have lots of time, and wish to use one
%% your platform's native fonts, see the commented out bits below for
%% XeTeX/XeLaTeX.  This is not for the faint at heart. 
%% (And shouldn't you be writing? :-)
%%

%% NFSS font specification (New Font Selection Scheme)
\usepackage{times,mathptmx,courier}
\usepackage[scaled=.92]{helvet}
\usepackage[framemethod=TikZ]{mdframed}
\mdfsetup{
  roundcorner=0pt,
  innerleftmargin=3pt,
  innertopmargin=0pt,
  innerrightmargin=0pt,
  innerbottommargin=0pt
}

%% Math or theory people may want to include the handy AMS macros
%\usepackage{amssymb}
%\usepackage{amsmath}
%\usepackage{amsfonts}

%% The pifont package provides access to the elements in the dingbat font.   
%% Use \ding{##} for a particular dingbat (see p7 of psnfss2e.pdf)
%%   Useful:
%%     51,52 different forms of a checkmark
%%     54,55,56 different forms of a cross (saltyre)
%%     172-181 are 1-10 in open circle (serif)
%%     182-191 are 1-10 black circle (serif)
%%     192-201 are 1-10 in open circle (sans serif)
%%     202-211 are 1-10 in black circle (sans serif)
%% \begin{dinglist}{##}\item... or dingautolist (which auto-increments)
%% to create a bullet list with the provided character.
\usepackage{pifont}

\usepackage{import}


%%%%%%%%%%%%%%%%%%%%%%%%%%%%%%%%%%%%%%%%%%%%%%%%%%%%%%%%%%%%%%%%%%%%%%
%% Configure fonts for XeTeX / XeLaTeX using the fontspec package.
%% Be sure to check out the fontspec documentation.
%\usepackage{fontspec,xltxtra,xunicode}	% required
%\defaultfontfeatures{Mapping=tex-text}	% recommended
%% Minion Pro and Myriad Pro are shipped with some versions of
%% Adobe Reader.  Adobe representatives have commented that these
%% fonts can be used outside of Adobe Reader.
%\setromanfont[Numbers=OldStyle]{Minion Pro}
%\setsansfont[Numbers=OldStyle,Scale=MatchLowercase]{Myriad Pro}
%\setmonofont[Scale=MatchLowercase]{Andale Mono}

%% Other alternatives:
%\setromanfont[Mapping=tex-text]{Adobe Caslon}
%\setsansfont[Scale=MatchLowercase]{Gill Sans}
%\setsansfont[Scale=MatchLowercase,Mapping=tex-text]{Futura}
%\setmonofont[Scale=MatchLowercase]{Andale Mono}
%\newfontfamily{\SYM}[Scale=0.9]{Zapf Dingbats}
%% END FONTS
%%%%%%%%%%%%%%%%%%%%%%%%%%%%%%%%%%%%%%%%%%%%%%%%%%%%%%%%%%%%%%%%%%%%%%
%%%%%%%%%%%%%%%%%%%%%%%%%%%%%%%%%%%%%%%%%%%%%%%%%%%%%%%%%%%%%%%%%%%%%%



%%%%%%%%%%%%%%%%%%%%%%%%%%%%%%%%%%%%%%%%%%%%%%%%%%%%%%%%%%%%%%%%%%%%%%
%%%%%%%%%%%%%%%%%%%%%%%%%%%%%%%%%%%%%%%%%%%%%%%%%%%%%%%%%%%%%%%%%%%%%%
%%
%% Recommended packages
%%
\usepackage{checkend}	% better error messages on left-open environments
\usepackage{graphicx}	% for incorporating external images

%% booktabs: provides some special commands for typesetting tables as used
%% in excellent journals.  Ignore the examples in the Lamport book!
\usepackage{booktabs}

%% listings: useful support for including source code listings, with
%% optional special keyword formatting.  The \lstset{} causes
%% the text to be typeset in a smaller sans serif font, with
%% proportional spacing.
\usepackage{listings}
\lstset{basicstyle=\sffamily\scriptsize,showstringspaces=false,fontadjust}

%% The acronym package provides support for defining acronyms, providing
%% their expansion when first used, and building glossaries.  See the
%% example in glossary.tex and the example usage throughout the example
%% document.
%% NOTE: to use \MakeTextLowercase in the \acsfont command below,
%%   we *must* use the `nohyperlinks' option -- it causes errors with
%%   hyperref otherwise.  See Section 5.2 in the ``LaTeX 2e for Class
%%   and Package Writers Guide'' (clsguide.pdf) for details.
\usepackage[printonlyused,nohyperlinks]{acronym}

%\usepackage[dvips]{graphicx}
% \usepackage{psfrag}
\usepackage{makeidx} 
\usepackage{amssymb,amsmath,color}

%  \usepackage[hang,small,bf]{caption}
%\usepackage{subfig}
%\usepackage[subfigure]{tocloft}
\usepackage{cite}

\usepackage{algorithmic}
\usepackage{epsfig} 
\usepackage{setspace}
\usepackage{enumerate}
\usepackage{url}
%\usepackage{natbib}
\usepackage[small,compact]{titlesec}
% \usepackage{subfigure}
\mathchardef\mhyphen="2D
\newcommand{\theadturn}[1]{%
\begin{turn}{90}\textbf{#1}\end{turn}
}
\newcommand{\thead}[1]{%
\textbf{#1}
}
\newcommand{\answer}[1]{%
\medskip\noindent\fbox{\parbox[l]{.97\hsize}{\emph{#1}}}\smallskip}

\newcommand{\footnoteremember}[2]{\footnote{~\scriptsize{#2}}
  \newcounter{#1}
  \setcounter{#1}{\value{footnote}}}
\newcommand{\footnoterecall}[1]{\footnotemark[\value{#1}]
}

\newcommand{\refFormula}[1]{(\ref{#1})}
\newcommand{\refFigure}[1]{Figure~\ref{#1}}
% \usepackage{rotating} 

\usepackage{bm}
\usepackage{url,moreverb,graphicx}
\usepackage{ifthen}
\usepackage{amssymb}
\usepackage{listings}
\usepackage{xspace}
\usepackage{rotating}
\usepackage{fixmath}
%\usepackage{relsize}
\usepackage{mdwlist}
%\usepackage{enumitem}
%\usepackage{paralist}
\usepackage{pseudocode}
\usepackage[lined,boxruled]{algorithm2e}
\usepackage{theorem}
\usepackage{amsmath}
\usepackage{multirow}
\usepackage{array}
% \let\labelindent\relax
\usepackage{enumitem}
%\usepackage{floatflt}
%\usepackage{caption}
%\usepackage{natbib}
\usepackage[font=scriptsize]{caption}
\usepackage{setspace}
\usepackage{lipsum}


%\usepackage{subfigure}
\usepackage{balance}

\usepackage{color}
\usepackage{textcomp}
\definecolor{lbcolor}{rgb}{0.9,0.9,0.9}
\lstset{
	%backgroundcolor=\color{lbcolor},
	tabsize=2,
	rulecolor=,
	language=html,
        basicstyle=\scriptsize,
        upquote=true,
        aboveskip={0.5\baselineskip},
        columns=fixed,
        showstringspaces=false,
        extendedchars=true,
        breaklines=true,
        prebreak = \raisebox{0ex}[0ex][0ex]{\ensuremath{\hookleftarrow}},
        frame=single,
        showtabs=false,
        showspaces=false,
        showstringspaces=false,
        identifierstyle=\ttfamily,
        keywordstyle=\color[rgb]{0,0,1},
        commentstyle=\color[rgb]{0.133,0.545,0.133},
        stringstyle=\color[rgb]{0.627,0.126,0.941},
        %emphstyle=\color[rgb]{0.827,0.126,0.941},
}


%Listings for JS
\definecolor{lightgray}{rgb}{.9,.9,.9}
\definecolor{darkgray}{rgb}{.4,.4,.4}
\definecolor{purple}{rgb}{0.65, 0.12, 0.82}
\definecolor{forestgreen}{rgb}{0.13, 0.55, 0.13}

\lstdefinelanguage{JavaScript}{
  keywords={typeof, new, true, false, catch, function, return, null, catch, switch, var, if, for, in, while, do, else, case, break, assertEquals, assertTrue,equal, ok},
  keywordstyle=\color{blue}\bfseries,
  ndkeywords={class, export, boolean, throw, implements, import, this, By, int,@Test},
  ndkeywordstyle=\color{darkgray}\bfseries,
  identifierstyle=\color{black},
  sensitive=false,
  comment=[l]{//},
  morecomment=[s]{/*}{*/},
  commentstyle=\color{forestgreen}\ttfamily,
  stringstyle=\color{purple}\ttfamily,
  morestring=[b]',
  morestring=[b]"
}
\lstset{
   language=JavaScript,
   %backgroundcolor=\color{lightgray},
   extendedchars=true,
   basicstyle=\scriptsize\ttfamily,
   showstringspaces=false,
   showspaces=false,
   numbers=left,
   numberstyle=\scriptsize,
   numbersep=-6pt,
   tabsize=2,
   breaklines=true,
   showtabs=false,
   captionpos=b,
   numberblanklines=false,
   escapeinside=\[\]}

%\textwidth 17.8cm
%\textheight 22.7cm
\setstretch{0.95}

%%Terminology
\newcommand{\rest}{\textsc{Rest}\xspace}
\newcommand{\jsf}{\textsc{Jsf}\xspace}
\newcommand{\ajax}{\textsc{Ajax}\xspace}
\newcommand{\headbf}[1]{\par\smallskip\noindent\textbf{#1.}}
\newcommand{\head}[1]{\subsubsection{#1}}
\newcommand{\headed}[1]{\noindent\textbf{#1}\ \ }
\newcommand{\webtwo}{\textit{Web~2.0}\xspace}
\newcommand{\crawljax}{\textsc{Crawljax}\xspace}
\newcommand{\petstore}{\textsc{PetStore}\xspace}
\newcommand{\etal}{et al.\xspace}
%\newcommand{\ie}{{i.e.,}\xspace}
%\newcommand{\eg}{{e.g.,}\xspace}
\newcommand{\atusatwo}{\textsc{Atusa~2.0}\xspace}
\newcommand{\tudu}{\textsc{TUDU}\xspace}
\newcommand{\todo}{\textsc{Todo}\xspace}
\newcommand{\taskfreak}{\textsc{TaskFreak}\xspace}
\newcommand{\tocview}{\textsc{TocView}\xspace}
\newcommand{\javascript}{{JavaScript}\xspace}
\newcommand{\artemis}{{Artemis}\xspace}
\newcommand{\jseft}{{JSEFT}\xspace}
\newcommand{\atrina}{{Atrina}\xspace}
\newcommand{\jscover}{{JSCover}\xspace}
\newcommand{\jquery}{\textsc{jQuery}\xspace}
\newcommand{\hitlist}{\textsc{HitList}\xspace}
\newcommand{\googlereader}{\textsc{Google Reader}\xspace}
\newcommand{\testsuite}[1]{\textsc{Test Suite #1}\xspace}
\newcommand{\version}[1]{\textsc{V#1}\xspace}
\newcommand{\proc}[2]{{\textbf{Procedure} \textsc{#1}(\textit{#2})}}
\newcommand{\junit}{\textsc{JUnit}\xspace}
\newcommand{\selenium}{\textsc{Selenium}\xspace}
\newcommand{\qunit}{\textsc{QUnit}\xspace}
\newcommand{\mutandis}{\textsc{Mutandis}\xspace}
\newcommand{\dustme}{\textsc{DMS}\xspace}
\newcommand{\helium}{\textsc{Helium}\xspace}
\newcommand{\cssess}{\textsc{CSSess}\xspace}
\newcommand{\jsart}{\textsc{JSart}\xspace}
\newcommand{\atusa}{\textsc{Atusa}\xspace}


\newcommand{\cssparser}{\textsc{CSSParser}\xspace}

\newcommand{\myspace}{\hspace{1mm}}


%% abbreviations and commands
\newcommand{\secref}[1]{Section~\ref{Sec:#1}}
\newcommand{\chapref}[1]{Chapter~\ref{Chap:#1}}
\newcommand{\figref}[1]{Figure~\ref{Fig:#1}}
\newcommand{\listref}[1]{Listing~\ref{List:#1}}
\newcommand{\tabref}[1]{Table~\ref{Table:#1}}
\newcommand{\algref}[1]{Algorithm~\ref{Alg:#1}}
\newcommand{\curl}[1]{\footnote{~\scriptsize\url{#1}}}
\newcommand{\fn}[1]{\footnote{~\scriptsize{#1}}}
\newcommand{\code}[1]{{\texttt{#1}}}
\newtheorem{mydef}{Definition}


\newcommand{\mycaption}[1]{%
\vspace{-1.3\baselineskip}
\caption{#1}
}

\lstloadlanguages{Java,XML,HTML}



%% The ubcdiss.cls loads the `textcase' package which provides commands
%% for upper-casing and lower-casing text.  The following causes
%% the acronym package to typeset acronyms in small-caps
%% as recommended by Bringhurst.


\renewcommand{\acsfont}[1]{{\scshape \MakeTextLowercase{#1}}}

%% color: add support for expressing colour models.  Grey can be used
%% to great effect to emphasize other parts of a graphic or text.
%% For an excellent set of examples, see Tufte's "Visual Display of
%% Quantitative Information" or "Envisioning Information".
\usepackage{color}
\definecolor{greytext}{gray}{0.5}

%% comment: provides a new {comment} environment: all text inside the
%% environment is ignored.
%%   \begin{comment} ignored text ... \end{comment}
\usepackage{comment}

%% The natbib package provides more sophisticated citing commands
%% such as \citeauthor{} to provide the author names of a work,
%% \citet{} to produce an author-and-reference citation,
%% \citep{} to produce a parenthetical citation.
%% We use \citeeg{} to provide examples
\usepackage[numbers,sort&compress]{natbib}
\newcommand{\citeeg}[1]{\citep[e.g.,][]{#1}}

%% The titlesec package provides commands to vary how chapter and
%% section titles are typeset.  The following uses more compact
%% spacings above and below the title.  The titleformat that follow
%% ensure chapter/section titles are set in singlespace.
\usepackage[compact]{titlesec}
\titleformat*{\section}{\singlespacing\raggedright\bfseries\Large}
\titleformat*{\subsection}{\singlespacing\raggedright\bfseries\large}
\titleformat*{\subsubsection}{\singlespacing\raggedright\bfseries}
\titleformat*{\paragraph}{\singlespacing\raggedright\itshape}

%% The caption package provides support for varying how table and
%% figure captions are typeset.
\usepackage[format=hang,indention=-1cm,labelfont={bf},margin=1em]{caption}

%% url: for typesetting URLs and smart(er) hyphenation.
%% \url{http://...} 
\usepackage{url}
\urlstyle{sf}	% typeset urls in sans-serif


%%%%%%%%%%%%%%%%%%%%%%%%%%%%%%%%%%%%%%%%%%%%%%%%%%%%%%%%%%%%%%%%%%%%%%
%%%%%%%%%%%%%%%%%%%%%%%%%%%%%%%%%%%%%%%%%%%%%%%%%%%%%%%%%%%%%%%%%%%%%%
%%
%% Possibly useful packages: you may need to explicitly install
%% these from CTAN if they aren't part of your distribution;
%% teTeX seems to ship with a smaller base than MikTeX and MacTeX.
%%
%\usepackage{pdfpages}	% insert pages from other PDF files
%\usepackage{longtable}	% provide tables spanning multiple pages
%\usepackage{chngpage}	% support changing the page widths on demand
%\usepackage{tabularx}	% an enhanced tabular environment

%% enumitem: support pausing and resuming enumerate environments.
%\usepackage{enumitem}

%% rotating: provides two environments, sidewaystable and sidewaysfigure,
%% for typesetting tables and figures in landscape mode.  
%\usepackage{rotating}

%% subfig: provides for including subfigures within a figure,
%% and includes being able to separately reference the subfigures.
\usepackage{subfig}

%% ragged2e: provides several new new commands \Centering, \RaggedLeft,
%% \RaggedRight and \justifying and new environments Center, FlushLeft,
%% FlushRight and justify, which set ragged text and are easily
%% configurable to allow hyphenation.
%\usepackage{ragged2e}

%% The ulem package provides a \sout{} for striking out text and
%% \xout for crossing out text.  The normalem and normalbf are
%% necessary as the package messes with the emphasis and bold fonts
%% otherwise.
%\usepackage[normalem,normalbf]{ulem}    % for \sout

%%%%%%%%%%%%%%%%%%%%%%%%%%%%%%%%%%%%%%%%%%%%%%%%%%%%%%%%%%%%%%%%%%%%%%
%% HYPERREF:
%% The hyperref package provides for embedding hyperlinks into your
%% document.  By default the table of contents, references, citations,
%% and footnotes are hyperlinked.
%%
%% Hyperref provides a very handy command for doing cross-references:
%% \autoref{}.  This is similar to \ref{} and \pageref{} except that
%% it automagically puts in the *type* of reference.  For example,
%% referencing a figure's label will put the text `Figure 3.4'.
%% And the text will be hyperlinked to the appropriate place in the
%% document.
%%
%% Generally hyperref should appear after most other packages

%% The following puts hyperlinks in very faint grey boxes.
%% The `pagebackref' causes the references in the bibliography to have
%% back-references to the citing page; `backref' puts the citing section
%% number.  See further below for other examples of using hyperref.
%% 2009/12/09: now use `linktocpage' (Jacek Kisynski): GPS now prefers
%%   that the ToC, LoF, LoT place the hyperlink on the page number,
%%   rather than the entry text.
\usepackage[bookmarks,bookmarksnumbered,%
    allbordercolors={0.8 0.8 0.8},%
    pagebackref,linktocpage%
    ]{hyperref}
%% The following change how the the back-references text is typeset in a
%% bibliography when `backref' or `pagebackref' are used
\renewcommand\backrefpagesname{\(\rightarrow\) pages}
\renewcommand\backref{\textcolor{greytext} \backrefpagesname\ }

%% The following uses most defaults, which causes hyperlinks to be
%% surrounded by colourful boxes; the colours are only visible in
%% PDFs and don't show up when printed:
%\usepackage[bookmarks,bookmarksnumbered]{hyperref}

%% The following disables the colourful boxes around hyperlinks.
%\usepackage[bookmarks,bookmarksnumbered,pdfborder={0 0 0}]{hyperref}

%% The following disables all hyperlinking, but still enabled use of
%% \autoref{}
%\usepackage[draft]{hyperref}

%% The following commands causes chapter and section references to
%% uppercase the part name.
\renewcommand{\chapterautorefname}{Chapter}
\renewcommand{\sectionautorefname}{Section}
\renewcommand{\subsectionautorefname}{Section}
\renewcommand{\subsubsectionautorefname}{Section}


%% If you have long page numbers (e.g., roman numbers in the 
%% preliminary pages for page 28 = xxviii), you might need to
%% uncomment the following and tweak the \@pnumwidth length
%% (default: 1.55em).  See the tocloft documentation at
%% http://www.ctan.org/tex-archive/macros/latex/contrib/tocloft/
% \makeatletter
% \renewcommand{\@pnumwidth}{3em}
% \makeatother

%%%%%%%%%%%%%%%%%%%%%%%%%%%%%%%%%%%%%%%%%%%%%%%%%%%%%%%%%%%%%%%%%%%%%%
%%%%%%%%%%%%%%%%%%%%%%%%%%%%%%%%%%%%%%%%%%%%%%%%%%%%%%%%%%%%%%%%%%%%%%
%%
%% Some special settings that controls how text is typeset
%%
% \raggedbottom		% pages don't have to line up nicely on the last line
% \sloppy		% be a bit more relaxed in inter-word spacing
% \clubpenalty=10000	% try harder to avoid orphans
% \widowpenalty=10000	% try harder to avoid widows
% \tolerance=1000

%% And include some of our own useful macros
% This file provides examples of some useful macros for typesetting
% dissertations.  None of the macros defined here are necessary beyond
% for the template documentation, so feel free to change, remove, and add
% your own definitions.
%
% We recommend that you define macros to separate the semantics
% of the things you write from how they are presented.  For example,
% you'll see definitions below for a macro \file{}: by using
% \file{} consistently in the text, we can change how filenames
% are typeset simply by changing the definition of \file{} in
% this file.
% 
%% The following is a directive for TeXShop to indicate the main file
%%!TEX root = diss.tex

\newcommand{\NA}{\textsc{n/a}}	% for "not applicable"
\newcommand{\eg}{e.g.,\ }	% proper form of examples (\eg a, b, c)
\newcommand{\ie}{i.e.,\ }	% proper form for that is (\ie a, b, c)
%\newcommand{\etal}{\emph{et al}}

% Some useful macros for typesetting terms.
\newcommand{\file}[1]{\texttt{#1}}
\newcommand{\class}[1]{\texttt{#1}}
\newcommand{\latexpackage}[1]{\href{http://www.ctan.org/macros/latex/contrib/#1}{\texttt{#1}}}
\newcommand{\latexmiscpackage}[1]{\href{http://www.ctan.org/macros/latex/contrib/misc/#1.sty}{\texttt{#1}}}
\newcommand{\env}[1]{\texttt{#1}}
\newcommand{\BibTeX}{Bib\TeX}

% Define a command \doi{} to typeset a digital object identifier (DOI).
% Note: if the following definition raise an error, then you likely
% have an ancient version of url.sty.  Either find a more recent version
% (3.1 or later work fine) and simply copy it into this directory,  or
% comment out the following two lines and uncomment the third.
\DeclareUrlCommand\DOI{}
\newcommand{\doi}[1]{\href{http://dx.doi.org/#1}{\DOI{doi:#1}}}
%\newcommand{\doi}[1]{\href{http://dx.doi.org/#1}{doi:#1}}

% Useful macro to reference an online document with a hyperlink
% as well with the URL explicitly listed in a footnote
% #1: the URL
% #2: the anchoring text
\newcommand{\webref}[2]{\href{#1}{#2}\footnote{\url{#1}}}

% epigraph is a nice environment for typesetting quotations
\makeatletter
\newenvironment{epigraph}{%
	\begin{flushright}
	\begin{minipage}{\columnwidth-0.75in}
	\begin{flushright}
	\@ifundefined{singlespacing}{}{\singlespacing}%
    }{
	\end{flushright}
	\end{minipage}
	\end{flushright}}
\makeatother

% \FIXME{} is a useful macro for noting things needing to be changed.
% The following definition will also output a warning to the console
\newcommand{\FIXME}[1]{\typeout{**FIXME** #1}\textbf{[FIXME: #1]}}

% END


%%%%%%%%%%%%%%%%%%%%%%%%%%%%%%%%%%%%%%%%%%%%%%%%%%%%%%%%%%%%%%%%%%%%%%
%%%%%%%%%%%%%%%%%%%%%%%%%%%%%%%%%%%%%%%%%%%%%%%%%%%%%%%%%%%%%%%%%%%%%%
%%
%% Document meta-data: be sure to also change the \hypersetup information
%%

\title{Effective Test Generation and Adequacy Assessment for JavaScript-based Web Applications}
%\subtitle{If you want a subtitle}

\author{Shabnam Mirshokraie}
\previousdegree{BSc. Computer Engineering,  Ferdowsi University of Mashhad, Iran, 2006}
\previousdegree{MSc. Computing Science, Simon Fraser University, Canada, 2010}

% What is this dissertation for?
\degreetitle{Doctor of Philosophy}

\institution{The University of British Columbia}
\campus{Vancouver}

\faculty{The Faculty of Applied Science}
\department{Electrical and Computer Engineering}
\submissionmonth{April}
\submissionyear{2015}

%% hyperref package provides support for embedding meta-data in .PDF
%% files
\hypersetup{
  pdftitle={Change this title!  (DRAFT: \today)},
  pdfauthor={Johnny Canuck},
  pdfkeywords={Your keywords here}
}

%%%%%%%%%%%%%%%%%%%%%%%%%%%%%%%%%%%%%%%%%%%%%%%%%%%%%%%%%%%%%%%%%%%%%%
%%%%%%%%%%%%%%%%%%%%%%%%%%%%%%%%%%%%%%%%%%%%%%%%%%%%%%%%%%%%%%%%%%%%%%
%% 
%% The document content
%%

%% LaTeX's \includeonly commands causes any uses of \include{} to only
%% include files that are in the list.  This is helpful to produce
%% subsets of your thesis (e.g., for committee members who want to see
%% the dissertation chapter by chapter).  It also saves time by 
%% avoiding reprocessing the entire file.
%\includeonly{intro,conclusions}
%\includeonly{discussion}

\begin{document}

%%%%%%%%%%%%%%%%%%%%%%%%%%%%%%%%%%%%%%%%%%%%%%%%%%
%% From Thesis Components: Tradtional Thesis
%% <http://www.grad.ubc.ca/current-students/dissertation-thesis-preparation/order-components>

% Preliminary Pages (numbered in lower case Roman numerals)
%    1. Title page (mandatory)
\maketitle

%    2. Abstract (mandatory - maximum 350 words)
%% The following is a directive for TeXShop to indicate the main file
%%!TEX root = diss.tex

\chapter{Abstract}
Today's modern Web applications rely heavily on \javascript and client-side run-time manipulation of the DOM (Document Object Model) tree. One way to provide assurance about the correctness of such highly evolving and dynamic applications is through testing. However, \javascript is loosely typed, dynamic, and notoriously challenging to analyze and test.

The work presented in this dissertation has focused on advancing the state-of-the-art in testing \javascript-based web applications by proposing a new set of techniques and tools. We proposed (1) a new automated technique for \javascript regression testing, which is based on inferring invariant assertions, (2) the first \javascript mutation testing tool, capable of guiding the mutation generation towards behaviour-affecting mutants in error-prone portions of the code, (3) an automatic technique to generate test cases for \javascript functions and events; Mutation analysis is used to generate test oracles, capable of detecting regression \javascript and DOM-level faults, and (4) utilizing existing DOM-dependent assertions as well as useful execution information inferred from a DOM-based test suite to automatically generate assertions for unit-level testing of \javascript functions.

To measure the effectiveness of the proposed approaches, we evaluated each method presented in this thesis by conducting various empirical studies and comparisons with existing testing techniques. The evaluation results point to the effectiveness of the proposed test generation and test assessment techniques in terms of accuracy and fault detection capability.


%This document provides brief instructions for using the \class{ubcdiss}
%class to write a \acs{UBC}-conformant dissertation in \LaTeX.  This
%document is itself written using the \class{ubcdiss} class and is
%intended to serve as an example of writing a dissertation in \LaTeX.
%This document has embedded \acp{URL} and is intended to be viewed
%using a computer-based \ac{PDF} reader.
%
%Note: Abstracts should generally try to avoid using acronyms.
%
%Note: at \ac{UBC}, both the \ac{GPS} Ph.D. defence programme and the
%Library's online submission system restricts abstracts to 350
%words.

% Consider placing version information if you circulate multiple drafts
%\vfill
%\begin{center}
%\begin{sf}
%\fbox{Revision: \today}
%\end{sf}
%\end{center}

\cleardoublepage

%    3. Preface
%%% The following is a directive for TeXShop to indicate the main file
%%!TEX root = diss.tex

\chapter{Preface}
During my PhD studies, I have conducted the research described in this dissertation in collaboration with my supervisors, Ali Mesbah and Karthik Pattabiraman at the University of British Columbia (UBC). 
Research projects included in this dissertation have been either published or currently under review. I was the main contributor for the research projects presented in each chapter, including the initial idea, developing, and evaluating the system. I had the collaboration of Ali Mesbah and Karthik Pattabiraman to discuss the projects and ideas, as well as making edits in the text. 

The following list presents publications for each chapter.
\begin{itemize}
\item \chapref{jsart}:
\begin{itemize}
\item ``\jsart: JavaScript Assertion-based Regression Testing" \cite{mirshokraie:icwe12},
S. Mirshokraie and A. Mesbah, International Conferencee on Web Engineering (ICWE), 2012, 238-252.
\end{itemize}
\item \chapref{mutandis}:
\begin{itemize}
\item ``Efficient JavaScript Mutation Testing" \cite{mirshokraie:icst13},
S. Mirshokraie, A. Mesbah and K. Pattabiraman, International Conference on Software Testing, Verification, and Validation (ICST), 2013, 74-83 (Best paper Runner-up award).
\item ``Guided Mutation Testing for JavaScript Web Applications" \cite{mirshokraie:tse15},
S. Mirshokraie, A. Mesbah and K. Pattabiraman, IEEE Transaction on Software Engineering (TSE), 2015, 429-444.
\end{itemize}
\item \chapref{jseft}:
\begin{itemize}
\item ``JSEFT: Automated JavaScript Unit Test Generation" \cite{mirshokraie:icst15},
S. Mirshokraie, A. Mesbah and K. Pattabiraman, International Conference on Software Testing, Verification, and Validation (ICST), 2015, 1-10 (Nominated for the best paper award).
\item ``PY\-THIA: Generating Test Cases with Oracles
for JavaScript Applications" \cite{shabnam:ase13},
S. Mirshokraie, A. Mesbah and K. Pattabiraman, Automated Software Engineering (ASE), 2013, New Ideas Track, 610-615.
%\item ``Unit Test Generation for JavaScript", S. Mirshokraie, A. Mesbah and K. Pattabiraman,
%Submitted to the Software Testing, Verification and Reliability (STVR) journal and is currently under review. 
\end{itemize} 
%\item \chapref{atrina}:
%\begin{itemize}
%\item ``Atrina: Inferring Unit Oracles from GUI Test Cases",
%Submitted to the International Conference on Software Testing, Verification, and Validation (ICST'16) and is currently under review.
%\end{itemize}
\end{itemize}

%At \ac{UBC}, a preface may be required.  Be sure to check the
%\ac{GPS} guidelines as they may have specific content to be included.

\cleardoublepage

%    4. Table of contents (mandatory - list all items in the preliminary pages
%    starting with the abstract, followed by chapter headings and
%    subheadings, bibliographies and appendices)
\tableofcontents
\cleardoublepage	% required by tocloft package

%    5. List of tables (mandatory if thesis has tables)
\listoftables
\cleardoublepage	% required by tocloft package

%    6. List of figures (mandatory if thesis has figures)
\listoffigures
\cleardoublepage	% required by tocloft package

%    7. List of illustrations (mandatory if thesis has illustrations)
%    8. Lists of symbols, abbreviations or other (optional)

%    9. Glossary (optional)
%%% The following is a directive for TeXShop to indicate the main file
%%!TEX root = diss.tex

\chapter{Glossary}

This glossary uses the handy \latexpackage{acroynym} package to automatically
maintain the glossary.  It uses the package's \texttt{printonlyused}
option to include only those acronyms explicitly referenced in the
\LaTeX\ source.

% use \acrodef to define an acronym, but no listing
\acrodef{UI}{user interface}
\acrodef{UBC}{University of British Columbia}

% The acronym environment will typeset only those acronyms that were
% *actually used* in the course of the document
\begin{acronym}[ANOVA]
\acro{ANOVA}[ANOVA]{Analysis of Variance\acroextra{, a set of
  statistical techniques to identify sources of variability between groups}}
\acro{API}{application programming interface}
\acro{CTAN}{\acroextra{The }Common \TeX\ Archive Network}
\acro{DOI}{Document Object Identifier\acroextra{ (see
    \url{http://doi.org})}}
\acro{GPS}[GPS]{Graduate and Postdoctoral Studies}
\acro{PDF}{Portable Document Format}
\acro{RCS}[RCS]{Revision control system\acroextra{, a software
    tool for tracking changes to a set of files}}
\acro{TLX}[TLX]{Task Load Index\acroextra{, an instrument for gauging
  the subjective mental workload experienced by a human in performing
  a task}}
\acro{UML}{Unified Modelling Language\acroextra{, a visual language
    for modelling the structure of software artefacts}}
\acro{URL}{Unique Resource Locator\acroextra{, used to describe a
    means for obtaining some resource on the world wide web}}
\acro{W3C}[W3C]{\acroextra{the }World Wide Web Consortium\acroextra{,
    the standards body for web technologies}}
\acro{XML}{Extensible Markup Language}
\end{acronym}

% You can also use \newacro{}{} to only define acronyms
% but without explictly creating a glossary
% 
% \newacro{ANOVA}[ANOVA]{Analysis of Variance\acroextra{, a set of
%   statistical techniques to identify sources of variability between groups.}}
% \newacro{API}[API]{application programming interface}
% \newacro{GOMS}[GOMS]{Goals, Operators, Methods, and Selection\acroextra{,
%   a framework for usability analysis.}}
% \newacro{TLX}[TLX]{Task Load Index\acroextra{, an instrument for gauging
%   the subjective mental workload experienced by a human in performing
%   a task.}}
% \newacro{UI}[UI]{user interface}
% \newacro{UML}[UML]{Unified Modelling Language}
% \newacro{W3C}[W3C]{World Wide Web Consortium}
% \newacro{XML}[XML]{Extensible Markup Language}
	% always input, since other macros may rely on it

\textspacing		% begin one-half or double spacing

%   10. Acknowledgements (optional)
%%% The following is a directive for TeXShop to indicate the main file
%%!TEX root = diss.tex

\chapter{Acknowledgments}
I would like to express my deepest gratitude to my senior supervisor, Ali Mesbah,
for mentoring me in the past four years with his patience and knowledge. Without his support, completion of this thesis would not have been possible for me.
I would also like to express my gratitude to Karthik Pattabiraman for the enriching feedback he provided during the conversations we had. The knowledge that I have gained through the insightful
discussions with Karthik are invaluable.

I am grateful to all the members at the Software Analysis and Testing Lab for providing me a stimulating and fun environment. Last but certainly not least, I owe my deepest gratitude to my
parents for their love and endless support. I will never forget what I owe them, and my eternal gratitude to them cannot be expressed by words. I deeply appreciate the never-ending patience I have received from my parents.




%   11. Dedication (optional)

% Body of Thesis (not all sections may apply)
\mainmatter

\acresetall	% reset all acronyms used so far


\chapter{Introduction} \label{chap:intro}
In this section, we provide an introduction to modern web application testing, followed by some of the current techniques used for automating the testing process.

\section{Test Automation} \label{Sec:web-testing}
Software testing is an integral part of the software engineering.
Testing helps to improve the quality and dependability of the applications.
However, software systems have become more complex in last decades, with using different technologies and programming languages, that are even implemented by different developers.
As a result, writing high-quality test cases for such applications that can assure their correct behaviour becomes more complicated, time consuming, and effort intensive for developers \cite{anand:jss13}.
Automatic test case generation can significantly reduce the time and manual effort, while
increasing the reliability of web applications.

Determining the desired behaviour of the application under test for a given input is called the Oracle Problem.
Manual testing is expensive and time consuming, mainly because of the manual effort that should be spent in identifying the proper oracle. Therefore, automating the oracle generation is an important part of the testing process. Our goal is to improve the dependability and reliability of the modern web applications by automating the test and oracle generation process with different techniques and tools proposed in this thesis.

\subsection{Web Testing}
One of the common engines of today's modern web applications is \javascript.
Developers employ \javascript to add functionality, dynamically change the GUI state,
and communicate with web servers. Given the increasing reliance of web applications on this language, it is important to check its correct behaviour.

Automatically generating effective test suites for \javascript applications is particularly challenging compared with traditional languages.
The event-driven and highly dynamic nature of \javascript make \javascript applications error-prone \cite{Ocariza:esem2013} and difficult to test.
Moreover, \javascript is used to seamlessly change the Document Object Model (DOM) at run-time.
DOM is an interface that allows programs to dynamically access and update the content, structure, and style of a document.
The dynamic interaction between \javascript and the DOM, as two separate components, can become quite complex cite{Ocariza:esem2013}. 

The huge event space of such applications hinders model inferring techniques to cover the whole state space in a limited amount of time. Moreover, inferring test assertions with high fault finding capability requires a precise analysis of the interaction between the \javascript and the DOM. 
Providing an accurate mapping between the two components becomes difficult as the \javascript code and the DOM increase in size. It is also challenging to identify a proper set of test oracles from the large number of oracles that potentially can be selected.

To test \javascript-based web applications, developers
often write test cases using frameworks such as \selenium \cite{selenium} to examine the correct interaction of the user with web application (GUI testing) and \qunit \cite{quint} to test the proper functionality of the individual units (unit testing).
Although such frameworks help to automate test execution, the
test cases need to be written manually, which can be tedious
and inefficient. 
Using \selenium to write DOM-based tests and assertions
requires little knowledge about the internal operations performed at the client side code; The tester needs only basic knowledge of common event sequences to cover important DOM elements to assert. 
This makes it easier for the tester to write DOM-based test suites. However, DOM-based assertions can potentially miss some portion of the
code, while more fine grained unit-level assertions might be capable of detecting such faults. Furthermore, since DOM-based tests are agnostic of the \javascript code, finding the root cause of an error during DOM-based testing is more expensive than during unit testing. 
\figref{domTestExample} shows a sample DOM-based test case. The test case contains the sequence of clicking on different DOM elements without an observable communication with the executed \javascript code.  
On the other hand,
writing unit test assertions for web applications that have rich interaction with the DOM through their \javascript code is more tedious. 
To generate unit-level assertions, the technique needs to precisely interpret the full range of interaction between the code level operations of a unit and the DOM level operations of a system, otherwise it may not be able to assert the correctness of a particular behaviour when the unit is used as a part of a system. The inherent characteristics of unit and DOM-based tests, indicate that they are complementary and that there is a trade-off in individually using each to detect faults. 
\begin{figure}
%\medskip
\begin{lstlisting}[language=Java]
	@Test
	public void testCase1(){
		WebElement divElem=driver.findElements(By.id("divElem"));
		divElem.click();
		driver.findElements(By.id("endCell")).getSize().height;
		WebElement startCell=driver.findElements(By.id("startCell"));
		startCell.click();
		driver.findElements(By.id("startCell"));
		...

	}
\end{lstlisting}
\vspace{-0.1in} 

\caption{\selenium test case.}
\label{Fig:domTestExample}
\vspace{-0.2in} 
\end{figure}
\subsection{Current Test and Oracle Generation Techniques} 
Different test and oracle generation techniques have been proposed to overcome the aforementioned problems.
\headbf{Web Test Generation} Marchetto and Tonella \cite{marchetto:search} propose a search-based algorithm for generating event-based sequences to test Ajax applications. 
Mesbah et al. \cite{mesbah:tse12}, use generic and application-specific invariants as a form of automated soft oracles for testing \ajax applications.
Sen \etal \cite{sen:fse13} propose a record and replay framework called Jalangi. It incorporates selective record-replay as well as shadow values and shadow execution to enable writing of heavy-weight dynamic analyses.
The framework is able to track generic faults such as \code{null} and \code{undefined} values as well as type inconsistencies in \javascript. 

Jensen \etal \cite{jensen:fse13} propose a technique to test the correctness of communication patterns between client and server in \ajax applications by incorporating server interface descriptions.
They construct server interface descriptions through an inference technique that can learn communication patterns from sample data.
Saxena \etal \cite{song:symb10} combine random test generation with the use of symbolic execution for systematically exploring a \javascript application's event space as well as its value space, for security testing.
Artzi \etal propose \artemis \cite{artzi:icse11}, which supports automated testing of \javascript applications.
\artemis considers the event-driven execution model of a \javascript application for feedback-directed testing.

\headbf{Oracle Generation} There has been limited work on oracle generation for testing. 
Fraser \etal \cite{fraser:tse12} propose $\mu$TE\-ST, which employs a mutant-based oracle generation technique.  It automatically generates unit tests for Java object-oriented classes by using a genetic algorithm to target mutations with high impact on the application's behaviour. They further identify~\cite{fraser:issta11} relevant pre-conditions on the test inputs and post-conditions on the outputs to ease human comprehension.
Staats \etal \cite{staats:icse12} address the problem of selecting oracle data,  which is formed as a subset of internal state variables as well as outputs for which the expected values are determined.
Mutation testing is applied to produce oracles and rank the inferred oracles in terms of their fault finding capability.
They merely focus on supporting the creation of test oracles by the programmer, rather than fully automating the process of test case generation.
Loyola \etal \cite{loyola:issta14} propose Dodona, which ranks program variables based on their dependencies during the program execution. Using this ranking, they suggest a set of variables to be monitored by the tester as assertion-based oracles. Dodona is also among the test oracle generation supporting systems.  

\headbf{Test Automation Challenges} Although, researchers have recently developed automated test generation techniques for \javascript-based applications \cite{artzi:icse11, marchetto:search, tonella:icst08, mesbah:tse12, song:symb10}, current techniques suffer from two main shortcomings, namely, they:
\begin{enumerate} 
%\vspace{-0.15in}
\item Target the generation of \emph{event sequences}, which operate at the event-level or DOM-level to cover the state space of the application. These techniques fail to capture faults that  do not propagate to an observable DOM state. As such, they potentially miss this portion of code-level \javascript faults. In order to capture such faults, effective test generation techniques need to target the code at the \javascript unit-level, in addition to the event-level.
\item Either ignore the oracle problem altogether or simplify it through generic \emph{soft oracles}, such as  W3C HTML  validation \cite{artzi:icse11,mesbah:tse12}, or  \javascript runtime exceptions \cite{artzi:icse11}.
A generated test case without assertions is not useful since code coverage alone is not the goal of software testing. For such generated test cases, the tester still needs to  manually write many assertions, which is time and effort intensive. 
On the other hand, soft oracles  target generic fault types and are limited in their fault finding capabilities.   %\cite{Richardson:icse92}. 
%While there has been some work on the generation of test inputs \cite{song:symb10},  
%Despite such limitations, the automatic creation of strong test oracles, \ie assertions, has not gained much attention. 
However, to be practically useful, unit testing requires strong oracles to determine whether the application under test executes correctly.
\item They merely focus on supporting the test oracle generation by the programmer. Thus, these approaches are not fully automating the process of test case creation \cite{staats:icse12,loyola:issta14}.
\end{enumerate}
To address the above mentioned shortcomings, we proposed a set of fully automated test case and assertion generation techniques for \javascript applications. Our techniques can capture error-prone dynamic interactions of \javascript with the DOM. We also generate unit-level tests, which enable us to find \javascript code-related faults.

\section{Adequacy Assessment} \label{Sec:adequacy}
While automated testing can help the tester to assure the application's dependability and detect faults in the application code, it does not reveal the trustworthiness of the written tests.
In order to understand how well the functionality and the data is being tested, we need to assess the quality of the tests.
A large body of research has been accomplished to assess the quality of test suites
from different perspectives: (1) code coverage, and (2) mutation analysis.
%\begin{description} [noitemsep, nolistsep]
%\item [Code coverage] which measures the degree to which the application's code is covered through a particular test suite.
%\item [Mutation analysis] which measures the effectiveness of a test suite in terms of its ability to detect injected faults.
%\end{description}
While code coverage tells how much of the source code is exercised by the test suite, it does not provide sufficient insight into the actual quality of the tests. Mutation testing has been proposed as a fault-based testing technique to assess and improve the quality of a test suite.

The main idea of mutation testing is to create mutants (i.e., modified versions of the program) and check if the test suite is effective at detecting the mutants. 
The technique first generates a set of mutants by applying a set of well-defined mutation operators on the original version of the system under test. 
These mutation operators typically represent subtle mistakes, such as typos, commonly made by programmers. A test suite's adequacy is then measured by its ability to detect (or `kill') the mutants, which is known as the adequacy score (or mutation score).

\subsection{Mutation Testing Challenges}
Despite being an effective test adequacy assessment method~\cite{andrews:icse05,jia:tse10}, mutation testing suffers from two main issues.  First, there is a high \emph{computational cost} in executing the test suite against a potentially large set of generated mutants. Second, there is a significant amount of effort  involved in distinguishing \emph{equivalent mutants}, which are syntactically different but semantically identical to the original program~\cite{budd:acta82}.  Equivalent mutants have no observable effect on the application's behaviour, and as a result, cannot be killed by test cases. Empirical studies indicate that about 45\% of all undetected mutants are equivalent~\cite{schuler:tvr12, madeyski:tse13}.
According to a recent study~\cite{madeyski:tse13}, it takes on average 15 minutes to manually assess one single first-order mutant. While detecting equivalent mutants consumes considerable amount of time, there is still no fully automated technique that is capable of detecting all the equivalent mutants \cite{madeyski:tse13}.

\headbf{Current Mutation Testing Approaches} A large body of research has been conducted to turn mutation testing into a practical approach.
To reduce the computational cost of  mutation testing, researchers have
proposed three main approaches to generate a smaller subset of all possible mutants: 
(1) \emph{mutant clustering} \cite{ji:seke09}, which is an approach that chooses a subset of
mutants using clustering algorithms; (2) \emph{selective mutation} \cite{barbosa:stvr01, siami:icse08, zhang:icse10}, which is based on a  
careful selection of more effective mutation operators to generate less mutants; and (3) \emph{higher order mutation} (HOM) testing \cite{jia:scam08}, which tries to find 
rare but valuable higher order mutants that denote subtle faults \cite{jia:tse10}.  

According to the taxonomy suggested by Madeyski \etal \cite{madeyski:tse13}, there are three main categories of approaches that address the problem of equivalent mutants: (1) detecting equivalent mutants, (2) avoiding equivalent mutant generation, and (3) suggesting equivalent mutants. As far as equivalent mutant detection techniques are concerned, the most effective approach is proposed by
Offutt and Pan \cite{offutt:tvr97, offutt:compass96}, which uses constraint
solving and path analysis. The results of their evaluation showed that the approach is able to detect on average the 45\% of the equivalent mutants. 
However, these solutions are involved with considerable amount of manual effort and are error-prone.

Among equivalent detection methods, program slicing has also been used in equivalent mutants detection \cite{hieron:tvr99}. %suggest slicing to assist
The goal there is to guide a tester in detecting the locations that are affected by a mutant.
Among avoiding equivalent mutant generation techniques, Dom\'inguez-Jim\'enez \etal \cite{dominguez:ist11} propose an evolutionary mutation testing
technique based on a genetic algorithm to cope with the high computational cost of mutation 
testing by reducing the number of mutants. Their system generates a strong subset
of mutants, which is composed of surviving and difficult to kill mutants. 
Their technique, however, cannot distinguish equivalent mutants from surviving non-equivalent mutants.
Bottaci \cite{bottaci:icstw10} presents a mutation analysis technique based on the available type information at run-time to avoid generating incompetent mutants. 
This approach is applicable for dynamically typed programs such as \javascript. 
However, the efficiency of the technique is unclear as they do not provide any empirical evaluation of their approach. 

Among the equivalent mutant suggestion techniques, Schuler \etal \cite{schuler:issta09} suggest possible equivalent mutants by checking invariant violations. They
generate multiple mutated versions and then classify the versions based on the number of violated invariants.
The system recommends testers to focus on those mutations that violate the most invariants.
In a follow-up paper \cite{schuler:tvr12}, they extend their work to assess the role of code coverage changes in detecting equivalent mutants. 

Deng \etal \cite{deng:icst13} implement a version of statement deletion (SDL) mutation operator for Java within the muJava mutation system. The design of SDL operator is based on a theory that performing mutation testing using only one mutation operator can result in generating effective tests. However, the authors cannot conclude that SDL-based mutation is as effective as selective mutation, which contains a sufficient set of mutation operators from all possible operators. Therefore, they only recommend for future mutation systems to include SDL as a choice. 

However, these solutions suffer from the following limitations:
\begin{enumerate}
\item They are involved with considerable amount of manual effort, and thus are error-prone;
\item The mutants need to be executed against the test suite, which limits the efficiency of the technique as the number of mutants increase.
\item The system only recommends testers to focus on those mutations that are more likely to be non-equivalent. These techniques are not fully automated and are used as a supporting system for the tester;  
\end{enumerate}
To tackle the above mentioned issues, we proposed a fully automated mutation generation technique that avoids generating equivalent mutants a priori by identifying behaviour-affecting portions of the code, and thus achieving greater efficiency. 
Our approach (1) reduces the number of equivalent mutants and (2) guides testers towards designing test cases for important portions of the code from the application's behaviour point of view.  
 
\section{Research Questions} \label{Sec:researchq}
To improve the dependability of \javascript web applications, we designed two high-level research questions: 

{\bf RQ 1.3.A.} \emph{How can we automatically generate effective test cases for \javascript applications?}

In response to web testing challenges, we (1) designed an automated test case and oracle generator, which is capable of detecting faults
in the \javascript applications for both unit and DOM level, and (2) proposed an approach to exploit the existing DOM-based test suite in order to generate unit-level assertions. 

{\bf RQ 1.3.B.} \emph{How can we effectively assess the quality of the existing \javascript test cases?}

To assess the quality of test cases, we proposed a new \javascript mutation testing approach, which helps to guide the
mutation generation process towards parts of the code that are error-prone or likely to influence the program's output.

\section{Contributions} \label{Sec:contrib}
In response to the first and second research questions as outlined in \secref{researchq}, the following papers have been published:
\begin{itemize}
\item \chapref{jsart}: ``JavaScript Assertion-based Regression Testing" \cite{mirshokraie:icwe12},
S. Mirshokraie and A. Mesbah, ICWE, 2012, 238-252.
\item \chapref{mutandis}: ``Efficient JavaScript Mutation Testing" \cite{mirshokraie:icst13},
S. Mirshokraie, A. Mesbah and K. Pattabiraman, ICST, 2013, 74-83 (Best paper Runner-up award);
``Guided Mutation Testing for JavaScript Web Applications" \cite{mirshokraie:tse15},
S. Mirshokraie, A. Mesbah and K. Pattabiraman, in press (TSE 2015);
\item \chapref{jseft}: ``JSEFT: Automated JavaScript Unit Test Generation" \cite{mirshokraie:icst15},
S. Mirshokraie, A. Mesbah and K. Pattabiraman, To appear (ICST 2015); ``PY\-THIA: Generating Test Cases with Oracles
for JavaScript Applications" \cite{shabnam:ase13},
S. Mirshokraie, A. Mesbah and K. Pattabiraman, ASE, 2013, New Ideas Track, 610-615.
\item \chapref{atrina}: ``Atrina: Inferring Unit Oracles from GUI Test Cases", under preparation.
\end{itemize}

I have also contributed to the following related publications:
\begin{itemize}
\item Automated Analysis of CSS Rules to Support Style Maintenance \cite{mesbah:icse12}: 
A. Mesbah and S. Mirshokraie, ICSE'12, 408-418;
\item A Systematic Mapping Study of Web Application Testing \cite{garousi:ist13}: 
V. Garousi, A. Mesbah, A. Betin Can and S. Mirshokraie, IST, vol. 55, no. 8, 1374-1396, 2013;
\end{itemize}

\subimport{jsart/}{jsart}
\subimport{mutandis/}{mutandis}
\subimport{jseft/}{jseft}
\subimport{atrina/}{atrina}
\chapter{Conclusion} \label{Chap:conc}
\javascript is increasingly being used to create modern interactive web applications that offload a considerable amount of their execution to the client-side. \javascript is a notoriously challenging language for web developers to use, maintain, analyze and test. This thesis has focused on exploring strategies for testing \javascript-based web applications. In accordance to the goal of this dissertation we designed two research questions:
\begin{itemize}
\item [RQ1] How can we generate effective test cases for \javascript web applications?
\item [RQ2] How can we effectively assess the quality of the written test suites for \javascript applications?
\end{itemize}
\section{Contributions}
%The main contributions of the thesis can be summarized as follows:
The main contributions of the thesis in response to the first research question (RQ1) are as follows: 
\begin{itemize}
\item A new automated technique for \javascript regression testing, which is based on dynamic analysis to infer invariant assertions; The obtained assertions are injected back into the \javascript code to uncover regression faults in subsequent revisions of the web application under test. 
\item An automatic technique to generate test cases for \javascript functions and events; We use a mutation-based algorithm to effectively generate test oracles, capable of detecting regression \javascript and DOM-level faults. The technique uses a combination of function converge maximization and function state abstraction algorithms to efficiently generate unit test cases.
\item Exploiting an existing DOM-based test suite to generate unit-level assertions for applications that highly interact with the DOM through the underlying \javascript code; We utilize
existing DOM-dependent assertions as well as useful execution information inferred from a DOM-based test suite to automatically generate assertions used for testing individual \javascript functions.
\end{itemize}
To address the second research question (RQ2), we made the following contribution:
\begin{itemize}
\item The first \javascript mutation testing tool, which is capable of guiding the mutation generation towards behaviour-affecting mutants in error-prone portions of the code; The mutation testing method combines dynamic and static analysis to mutate branches that are within highly ranked functions and exhibit high structural complexity.
\end{itemize}

%    3. Notes
%    4. Footnotes

%    5. Bibliography
\begin{singlespace}
\raggedright
\bibliographystyle{abbrvnat}
\bibliography{../biblio}
\end{singlespace}

\appendix
%    6. Appendices (including copies of all required UBC Research
%       Ethics Board's Certificates of Approval)
%\include{reb-coa}	% pdfpages is useful here
%\chapter{Supporting Materials}

This would be any supporting material not central to the dissertation.
For example:
\begin{itemize}
\item additional details of methodology and/or data;
\item diagrams of specialized equipment developed.;
\item copies of questionnaires and survey instruments.
\end{itemize}


\backmatter
%    7. Index
% See the makeindex package: the following page provides a quick overview
% <http://www.image.ufl.edu/help/latex/latex_indexes.shtml>


\end{document}
