\subsection{Discussion} \label{Sec:discussion}
% Moved time efficiency to results
%Moreover, this reiterates the known performance shortcomings of approaches that rely on mutant generation.
\headbf{Fault Masking} As we mentioned in \secref{explicitAssertions}, the concrete value of an entity in the computed backward slice can potentially be used as the expected value of the entity in explicit assertions to test the current version of the application.
The actual values of the related entities in the backward slice are correct unless there exists a masked fault which is concealed in the chain of computations and thus does not propagate to the checked state of the DOM element. However, we conjecture that fault masking rarely happens in \javascript web applications as it is more prevalent in programs with many small expressions whose results are stored in several intermediate values. We also observed no fault masking occurrence during the evaluation of \atrina on seven \javascript applications used in this study.
\headbf{Limitations} The effectiveness of the generated assertions by \atrina in terms of fault finding capability depends on the quality of human-written DOM-based test cases. If the DOM assertions contained in the DOM-based test suite check irrelevant information, the explicit assertions obtained by our tool will point to entities that may not be important from the tester's point of view. This can also negatively affect the fault finding capability of implicit assertions as they are indirectly inferred from the DOM-based assertions. Moreover, if the human-written test suite does not execute application's state with effective DOM elements, our tool is not able to infer effective candidate assertions.   