\section{Conclusions}
%Mutation testing systematically evaluates the quality of existing tests suites.
%However, mutation testing suffers from equivalent mutants, as well as a high computational cost associated with a large pool of generated mutants.
In this work, we proposed a guided mutation testing technique that leverages dynamic and static characteristics of the system under test to selectively mutate portions of the code that exhibit a high probability of (1) being error-prone, or (2) affecting the observable behaviour of the system, and thus being non-equivalent. Thus, our technique is able to minimize the number of generated mutants while increasing their effect on the semantics of the system. 
We also proposed a set of \javascript-specific mutation operators that mimic developer mistakes in practice.
We implemented our approach in an open source mutation testing tool for \javascript, called \mutandis.  
The evaluation of \mutandis points to the efficacy of the approach in generating behaviour-affecting mutants.

%Given the growing popularity of \javascript and the challenges of testing this dynamic language,  we see many opportunities for using \codename in practice. Further, researchers working in the area can use \codename to assess and compare the adequacy of their web application testing techniques.

%Our future work will include comparing different heuristics for ranking the importance of functions and variables as well as exploring ways to rank branches for mutation testing, in addition to ranking functions and variables.

%Given the growing popularity of \javascript and the challenges of testing this dynamic language,  we see many opportunities for using \codename in practice. Further, researchers working in the area can use \codename to assess and compare the adequacy of their web application testing techniques.

%- many testing tools are coming out, no automatic way of assessing their fault-finding capabilities; \codename can be used by web application testing tools to assess their adequacy. 


