%% The following is a directive for TeXShop to indicate the main file
%%!TEX root = diss.tex

\chapter{Abstract}
Today's modern Web applications rely heavily on \javascript and client-side run-time manipulation of the DOM (Document Object Model) tree. One way to provide assurance about the correctness of such highly evolving and dynamic applications is through testing. However, \javascript is loosely typed, dynamic, and notoriously challenging to analyze and test.

The work presented in this dissertation has focused on advancing the state-of-the-art in testing \javascript-based web applications by proposing a new set of techniques and tools. We proposed (1) a new automated technique for \javascript regression testing, which is based on inferring invariant assertions, (2) the first \javascript mutation testing tool, capable of guiding the mutation generation towards behaviour-affecting mutants in error-prone portions of the code, (3) an automatic technique to generate test cases for \javascript functions and events; Mutation analysis is used to generate test oracles, capable of detecting regression \javascript and DOM-level faults, and (4) utilizing existing DOM-dependent assertions as well as useful execution information inferred from a DOM-based test suite to automatically generate assertions for unit-level testing of \javascript functions.

To measure the effectiveness of the proposed approaches, we evaluated each method presented in this thesis by conducting various empirical studies and comparisons with existing testing techniques. The evaluation results point to the effectiveness of the proposed test generation and test assessment techniques in terms of accuracy and fault detection capability.


%This document provides brief instructions for using the \class{ubcdiss}
%class to write a \acs{UBC}-conformant dissertation in \LaTeX.  This
%document is itself written using the \class{ubcdiss} class and is
%intended to serve as an example of writing a dissertation in \LaTeX.
%This document has embedded \acp{URL} and is intended to be viewed
%using a computer-based \ac{PDF} reader.
%
%Note: Abstracts should generally try to avoid using acronyms.
%
%Note: at \ac{UBC}, both the \ac{GPS} Ph.D. defence programme and the
%Library's online submission system restricts abstracts to 350
%words.

% Consider placing version information if you circulate multiple drafts
%\vfill
%\begin{center}
%\begin{sf}
%\fbox{Revision: \today}
%\end{sf}
%\end{center}
