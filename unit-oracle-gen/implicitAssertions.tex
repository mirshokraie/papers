\subsubsection{Implicit Assertions} \label{Sec:implicitAssertions}
To this end we gather all the statements that explicitly affect the computations relevant to a given DOM-based assertion. While assertions inferred from such statements are inherently important, we further need to consider entities that are implicitly influenced by the checked DOM element in the manually-written test suite. For this purpose we apply a dynamic forward slice on the statements collected from a backward slice of a DOM-based assertion. A forward slice with respect to a statement $st$,
indicates how subsequently a value computed at $st$ is being used. This can help the tester to ensure that $st$ properly establishes the expected outcome of the computations assumed by the later statements. 
Given the importance of statements involved in code-level computations of a DOM-based assertion, using forward slice is useful to check that there are no unforeseen effects on the application's behavior by a modification to such statements. 

Dynamic forward slice is performed on the subset of code statements which is previously instrumented as explained in \secref{domToCode}. The process of forwards slicing is similar to the backwards slicing technique as discussed earlier (\secref{domToCode}). The only difference is that it is performed in a forward direction. The slicing criterion of the forward slice module is either a variable, object's property, or an accessed DOM property extracted from the statements in a backward slice. The accessible entities, which have been set within the collected forward slice statements establish our implicit assertions.   