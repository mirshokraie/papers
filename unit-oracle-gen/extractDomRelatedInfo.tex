\subsection{Extracting DOM-Related Characteristics} \label{Sec:extractDomRelatedInfo}
The DOM connects a test case to the web application's \javascript code. Therefore, we first need to analyze the DOM-based test suite and extract the following pieces of information: (1) DOM-related operations of the existing test suite that may have tight connection with the \javascript code, and (2) accessed DOM properties, which are potentially important from the application's point of view, however, they are left unchecked in the manually-written test suite. In the following we describe each in details.
\headbf{DOM-Related Operations}
Any generated test case requires to check the correctness of the application's behaviour. In a DOM-based test case the expected behaviour is verified through DOM-based assertions. 
DOM assertions play a significant role in our approach as they can guide us towards important portions of the underlying \javascript code that need to be checked in unit-level assertions.

DOM assertion is defined as $<DOMProp,ExpVal>$, where $DOMProp$ is a DOM element feature (e.g. $text$, $size$, DOM attributes), and $ExpVal$ is the correct value expected by the assertion. Through the rest of the paper, we call DOM element features as DOM properties.
For each DOM assertion we record the accessed DOM properties as well as the accessed DOM elements in the test case pertaining to the assertion. In order to link DOM assertions to the relevant \javascript code we further need to track DOM's evolution as the test case runs. We make use of \code{document.onload} event to log the initial DOM state. An observer module is then used to monitor any mutations on the DOM during the test case execution. In addition to DOM mutations, we also keep track of \javascript events as well as invoked event handlers. These information is later used to find the initial point of contact between a DOM mutation and the executed code segments. 
%test suit instrumentation
\headbf{Candidate DOM Elements}
In addition to DOM-based assertions, we further consider DOM element properties, that are frequently accessed within the application as the test case runs. The intuition is that frequent use of a given DOM property can point to the extent of application's behaviour dependency on the DOM property. Thus, if changes happen to that property through the \javascript code, it is important to assert the correctness of such mutations. We define the access frequency of a DOM element property as the number of times that the element's property has been read during the execution of a test case.       
%application instrumentaion native event wrapping    
