\section{Empirical Evaluation} \label{Sec:evaluation}
To quantitatively assess the efficacy of our test generation approach, we have conducted a case study, in which we address the following research questions:

\begin{description}[noitemsep]
\item [RQ1] Is \tool more effective than DOM-based assertions written manually by the tester in terms of fault finding capability?
\item [RQ2] How accurate is our tool in mapping DOM-based assertions to the corresponding \javascript code?
\item [RQ3] How does \tool compare to the existing mutation-based approach for identifying unit test assertions?
\end{description}

\tool and the experimental data are available for download \cite{atrina-dl}.
\subsection{Objects}
Our study includes four open source \javascript web applications that have \selenium test cases. \tabref{objectsTable} presents the experimental objects and their properties. Phormer \cite{phormer} is a photo gallery web application. EnterpriseStore \cite{enterpriseStore} is an asset management web application. WolfCMS \cite{wolfcms} is a content management system, and Claroline \cite{claroline} is a collaborative online learning and course management system. 
\begin{table}
        \caption{Characteristics of the experimental objects.} \label{Table:objectsTable}        
{\scriptsize
\centering
%    \begin{center}
       
      %  \subtable[Experimental subjects and the corresponding exploration data]
            {
           \begin{tabular}{l|l|l|>{\centering}m{1cm}|c} \hline
\thead{ID} &\thead{Name} &\thead{LOC (JS)} &\thead{\# Test Cases} &\thead{\# Assertions}  \\  \hline 

\hline

1  & Phormer & 1.5K & 7 & 18    \\ \hline
           
2 & EnterpriseStore & 57K & 19 & 21  \\ \hline

3 & WolfCMS & 1.3K & 12 & 42  \\ \hline

4 & Claroline & 36K & 23 & 35 \\ \hline

5 & StudyRoom & 10.6K & 12 & 23 \\ \hline

6 & AddressBook & 1.1K & 13 & 14 \\ \hline

7 & Brotherhood & 0.8K & 10 & 10 \\ \hline
\end{tabular}
            }

%\end{center}
}
\vspace{-0.2in} 
\end{table}
\subsection{Setup} \label{Sec:setup}
To address our research questions, we provide the URL as well as the available manually written DOM-based test suite of each experimental object to \tool. Unit level test assertions are then automatically generated by the tool.

\headbf{Comparison with human-written DOM-based Assertions (RQ1)} To assess the effectiveness of the \tool, we compare the human written DOM-based assertions with the unit-level test assertions generated by our approach in terms of fault finding capability.
The experimental objects do not come with a rich version history to apply \tool on real regression changes. Therefore we mimic regression faults by injecting mutations to the application, and evaluate the tool's ability in detecting the seeded faults. We randomly inject 50 first-order mutations into the \javascript code of the applications. The mutations represent common mistakes made by developers when developing a given web application \cite{mirshokraie:icst13}, e.g., changing the value of a variable or modifying a conditional statement, altering unary operations, changing the ID/tag name passed into DOM access functions such as \code{getElementById} or \code{getElementsByTagName}, and modifying the attribute name/value in \code{setAttribute}. The fault is considered detected if an assertion generated by \tool fails and our manual examination confirms that the failed assertion is detecting the seeded fault.
\headbf{Accuracy (RQ2)} To evaluate the accuracy of \tool, we measure precision and recall. We manually compare the slices generated by \tool with the \javascript lines of code, which are considered relevant to each assertion. Precision and recall are defined as follows:
\begin{description}[noitemsep, nolistsep, font=\normalfont\itshape]
\item [Precision] is the fraction of lines in a slice produced by \tool, that are actually related to the human-written DOM-based assertion. 
\item [Recall] is the fraction of the correct set of related lines of code to each assertion, which is actually present in the slice produced by \tool.
\end{description}
\headbf{Comparison with Mutation-based Assertion Generation (RQ3)}

