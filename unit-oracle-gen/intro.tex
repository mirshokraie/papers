\section{Introduction} \label{Sec:intro}
A large body of research on automated testing of \javascript web applications has focused on automatically generating test input data to fulfill the coverage criteria, however, coverage alone is not the goal of software testing.
Current test generation techniques for \javascript applications are limited in the following categories:
\begin{itemize}
\item They rely on soft oracles such as HTML validation, or runtime exceptions \cite{artzi:icse11}. These oracles fail to capture logical and computational errors. 
\item They target DOM-based test cases. Our previous findings \cite{mirshokraie:icst15} indicate that DOM-based assertions can potentially miss the related portion of
code-level failure, while a more fine grained unit-level assertions are capable of detecting such faults.
\item They generate test oracles based on mutation testing techniques \cite{mirshokraie:icst15, fraser:tse12}. Mutation-based approaches suffer from computational cost and equivalent mutant problem.
\end{itemize}
%In this work we focus on generating unit test oracles at the \javascript code-level for  web applications that have rich interaction with the DOM through their \javascript code.
Comparing to \javascript libraries that do not have business logic or direct knowledge of the relationship between the DOM, CSS, and the \javascript, generating unit test oracles  
at \javascript code-level for web applications that have rich interaction with the DOM through their \javascript code is more tedious. To write unit-level assertions for such web applications, the tester needs to precisely understand the interaction between the DOM and \javascript code, which is responsible for updating DOM elements. On the other hand,
writing DOM-based test and assertion using frameworks such as Selenium
requires little knowledge about the underlying client side code. Moreover, the tester may have basic knowledge of common event sequences to cover or important DOM elements to assert. This makes it easier for the tester to write DOM-based test suite.
In this work, we propose to exploit existing DOM-based test suite to generate unit-level assertions at the \javascript code-level for those applications that highly interact with the DOM through the underlying \javascript code. We utilize
existing DOM-dependent assertions as well as useful execution information inferred from a DOM-based test suite to automatically generate assertions used for testing individual \javascript functions. The main contributions of our work include:
\begin{itemize}
\item A slicing-based technique to generate unit-level assertions capable of testing \javascript functions by utilizing existing DOM-based test assertions.
\item A method for detecting unchecked important DOM properties with ties to the \javascript code.   
\item An implementation of our approach in a tool, called \tool. 
\item An empirical evaluation to assess the efficacy of \tool.
\end{itemize}
    






