\section{Introduction} \label{Sec:intro}
\javascript is extensively used in today's modern web applications. The close relation between the Document Object Model (DOM) and the underlying \javascript code creates an interactive web application. To check the application's behaviour from an end-user's perspective, testers often use popular frameworks such as Selenium. Using these frameworks to write DOM-based tests and assertions
requires little knowledge about the underlying client side code. Moreover, the tester may have basic knowledge of common event sequences to cover or important DOM elements to assert. This makes it easier for the tester to write DOM-based test suites. On the other hand,
generating unit test assertions at \javascript code-level for web applications that have rich interaction with the DOM through their \javascript code is more tedious. To write unit-level assertions, the tester needs to precisely understand the interaction between the DOM and the \javascript code, which is responsible for updating DOM elements. 

In general current test generation techniques for \javascript applications are limited in the following ways:
\begin{itemize}[noitemsep]
\item They rely on soft oracles such as HTML validation, or runtime exceptions \cite{artzi:icse11}. These oracles fail to capture logical and computational errors. 
\item They target DOM-based test cases. Our previous findings \cite{mirshokraie:icst15} indicate that DOM-based assertions can potentially miss the related portion of
code-level failure, while more fine grained unit-level assertions are capable of detecting such faults.
\item They generate test oracles based on mutation testing techniques \cite{mirshokraie:icst15, fraser:tse12}. Mutation-based approaches suffer from high computational cost and equivalent mutants which are syntactically different but semantically are the same as the original application. 
\end{itemize}
%In this work we focus on generating unit test oracles at the \javascript code-level for  web applications that have rich interaction with the DOM through their \javascript code.
%Comparing to \javascript libraries that do not have business logic or direct knowledge of the relationship between the DOM, CSS, and the \javascript, generating unit test oracles  
%at \javascript code-level for web applications that have rich interaction with the DOM through their \javascript code is more tedious. To write unit-level assertions for such web applications, the tester needs to precisely understand the interaction between the DOM and \javascript code, which is responsible for updating DOM elements. On the other hand,
%writing DOM-based test and assertion using frameworks such as Selenium
%requires little knowledge about the underlying client side code. Moreover, the tester may have basic knowledge of common event sequences to cover or important DOM elements to assert. This makes it easier for the tester to write DOM-based test suite.
Recently, Milani Fard \etal propose using DOM-based test suite of the web application to regenerate assertions for new detected states through exploring alternative paths of the application. However, the new assertions generated by this technique remain at DOM-level without considering the relation between the \javascript code and the DOM.

In this work, we propose to exploit existing DOM-based test suite to generate unit-level assertions at the \javascript code-level for those applications that highly interact with the DOM through the underlying \javascript code. We utilize
existing DOM-dependent assertions as well as useful execution information inferred from a DOM-based test suite to automatically generate assertions used for testing individual \javascript functions. To the best of our knowledge this work is the first to propose an approach for generating unit-level assertions through analyzing the relation between the \javascript code and the DOM within the existing DOM-based test suite of the application. The main contributions of our work include:
\begin{itemize}[noitemsep]
\item A slicing-based technique to generate unit-level assertions capable of testing \javascript functions by utilizing existing DOM-based test assertions.
\item A technique for selecting important DOM elements with ties to the \javascript code, which remain unchecked in the existing DOM-based test suite.
\item An implementation of our approach in a tool, called \tool. 
\item An empirical evaluation to assess the efficacy of \tool on four open-source web applications.
\end{itemize}
    






