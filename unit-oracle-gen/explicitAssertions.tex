\subsubsection{Explicit Assertions} \label{Sec:explicitAssertions}
After collecting all the statements, that are relevant to a given DOM-based assertion, we extract accessible entities from these statements (\textsc{Accessibles} in line 23 of the algorithm).
Type of accessible entities includes (1) the function's returned value, (2) the used global variables in that function, (3) the object's property where the object is accessible in the outer scope of the function, and/or (4) the accessed DOM element in that function. Dynamic backward slice of a DOM-based assertion helps to (1) track all statements that contribute to the checked result and as such identify those entities that might have influenced the checked property value of the DOM element, and (2) eliminate unrelated entities which are not involved in the computation that leads to the update performed on the checked DOM element.

Since our dynamic slice is extracted from the program run, we can track all concrete values associated with accessible entities.
During the run of a test case, there might be different instances where a given statement is executed. Different execution instances can lead to different behaviour. Since we are employing dynamic slicing, an instance that leads to the required behaviour, which is verified through the DOM-based assertion, is on the backward slice. Given that the manually-written expected value, that is checked against the DOM's property is valid, the concrete values of related entities in the backward slice are potentially correct unless there exists a masked fault which is concealed in the chain of computations and thus does not propagate to the checked state of the DOM element. We conjecture that fault masking rarely happens in \javascript web applications as it is more prevalent in programs with many small expressions whose results are stored in several intermediate values. Therefore, concrete value of an entity in the backward slice can potentially be used as the expected value of the entity in unit-level assertions to test the current version of the application.
$explicitAsstn$ in line 23 of \algref{algorithm} contains the inferred explicit assertions.
In our running example (\figref{example}), explicit assertions check the correctness of \code{customer.payable}, \code{coupon.expired}, as well as \code{price} and \code{quantity} properties which belong to \code{selItem} object.  
Assuming that the original price of the item is 100, the number of selected item is 1, and the calculated discount according to the \code{value} attribute of a DOM element with ID \code{couponButt} is 30, then the expected values included in the assertions for each of the entities are 70, boolean value \code{true}, 100, and 1, respectively.  

%We compare the value of the entity immediately after the relevant statement is executed in the backward slice with the entity's value before the function exits. If value of the entity remains the same, we use it towards the expected value in the unit-level assertion.  
%However, if the entity pertaining to a DOM change is reassigned in the code after the DOM gets updated and before the function exits, then the concrete value of the entity can be used for the purpose of regression testing unless the tester provides the proper expected value.