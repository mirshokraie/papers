\section{Related Work} \label{Sec:related}
While automated test generation has significantly addressed in the literature, there has been limited work on supporting the construction test oracles.
Xie \etal explore test oracle generation for GUI systems \cite{xie:tosem07}. 
Eclat \cite{pacheco:ecoop05}, and DiffGen \cite{taneja:ase08} are used for automatically generating invariant-based oracles. 
Fraser \etal \cite{fraser:tse12} propose a mutation-based oracle generation system called $\mu$TE\-ST. $\mu$TE\-ST automatically generates unit tests for Java object-oriented classes by employing a genetic algorithm which target mutations with high impact on the application's behaviour. They further enhance the system \cite{fraser:issta11} to improve human comprehension through identifying relevant pre-conditions on the test inputs and post-conditions on the outputs. The authors assume that the tester will manually correct the generated oracles. However, the results on the effectiveness of such approaches which rely on the "generate-and-fix" assumption to construct test oracles are not conclusive \cite{fraser:issta13}

Staats \etal \cite{staats:icse11} propose an oracle data selection technique, which is based on mutation testing to produce oracles and rank the inferred oracles in terms of their fault finding capability. This work suffers from the scalability issues of mutant-generation based techniques as well as the problem of estimating the proper number of mutants required for generating effective oracle data set.
Similar to mutation-based techniques, differential test case generation approaches \cite{taneja:ase08, elbaum:tse09} also target generating test cases that show the difference between two versions of a program. However, mutation-based techniques do not require two different versions of the application since the generated differences are in the form of controllable mutations to detect regression faults in future versions of the program.
Pastore \etal \cite{pastore:icst13} exploit crowd sourcing approach to check assertions. In this approach the developer produces tests and provides sufficient API documentation for the crowd such that crowd workers can determine the correctness of assertions. However, recruiting qualified crowd to generate test oracles can be quite challenging.
  


